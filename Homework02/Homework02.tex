\documentclass[12pt,letterpaper,titlepage]{article}

\usepackage{fontspec}
\defaultfontfeatures{Mapping=tex-text}
\usepackage{xunicode}
\usepackage{xltxtra}
\usepackage{amsmath}
\usepackage{pdfpages}
\usepackage{amsfonts}
\usepackage{bbold}
\usepackage{amssymb}
\setcounter{secnumdepth}{0}
\usepackage{nameref}
\usepackage{enumitem}
\usepackage{environ}
\usepackage{pgfplots}
\usepackage{listings}

\showboxdepth=\maxdimen
\showboxbreadth=\maxdimen


\usepackage{paracol}
\usepackage{wrapfig}
\globalcounter{table}
\globalcounter{figure}
\usepackage{graphicx}
\usepackage[left=1in,right=1in,top=1in,bottom=1in]{geometry}
\graphicspath{{img/}}

\author{Jacob Abel}
\title{	Homework 2
	\\\large ECE2500 CRN:82943
}

\setlength{\parskip}{0.5em}

\begin{document}
\maketitle
\begin{raggedright}

\paragraph{Problem 1: } Extend each of the 16-bit numbers given below into 32-bit numbers using both the ZeroExtImm and SignExtImm schemes. Write your answers in hex.

\begin{enumerate}
\item 0xC943
\begin{description}
\item[ZeroExtImm:] 0x0000C943
\item[SignExtImm:] 0xFFFFC943
\end{description}
\item 0x3A04
\begin{description}
\item[ZeroExtImm:] 0x00003A04
\item[SignExtImm:] 0x00003A04
\end{description}
\end{enumerate}

\paragraph{Problem 2: }
The C language permits a variable to be left- or right-shifted by a non-constant amount (e.g. i >> j where i and j are variables), but the MIPS instruction set only supports shifts by a constant value (e.g. i >> 2). In words rather than code, describe how a variable-length right shift could be performed using MIPS instructions.

To implement a variable length right shift using MIPS instructions, you can create a loop over the shift length j of a right shift operation on i. 

\paragraph{Problem 3: }
Provide the instruction format type (R/I/J) and 32-bit hexadecimal encoding for each of the following instructions:

\begin{enumerate}
\item \begin{lstlisting}
add $s1, $t0, $t1
      \end{lstlisting}
\begin{description}
\item[Format:] R
\item[32-bit hex encoding:] (0x00, 0x08, 0x09, 0x11, 0x00, 0x20) 

-> 000000 01000 01001 10001 00000 100000 

-> 0000 0001 0000 1001 1000 1000 0010 0000 

-> 0x02284820
\end{description}

\item \begin{lstlisting}
lw $t2, 32($s0)
      \end{lstlisting}
\begin{description}
\item[Format:] I
\item[32-bit hex encoding:] (0x23, 0x10, 0x0A, 0x10) 

-> 100011 10000 01010 0000000000010000

-> 1000 1110 0000 1010 0000 0000 0001 0000

-> 0x8B0A0010
\end{description}
\end{enumerate}

\clearpage

\paragraph{Problem 4: }
Provide the type (R/I/J) and MIPS assembly language instruction for the instructions described by the following MIPS fields:

\begin{enumerate}
\item \begin{lstlisting}
op=0x0, rs=0x10, rt=0x11, rd=0x08, shamt=0x0, funct=0x27
\end{lstlisting}
\begin{description}
\item[Format:] R
\item[MIPS Instruction:] 
\begin{lstlisting}
nor $t0, $s0, $s1
\end{lstlisting}
\end{description}

\item \begin{lstlisting}
op=0xB, rs=0x18, rt=0x2, immediate=0xF21D
\end{lstlisting}
\begin{description}
\item[Format:] I
\item[MIPS Instruction:] 
\begin{lstlisting}
sltiu $v0, $t8, 61981
\end{lstlisting}
\end{description}
\end{enumerate}

\paragraph{Problem 5: }	
Provide the type (R/I/J) and MIPS assembly language instruction for the following machine code: 00000001000010010101000000100011

-> 00000001000010010101000000100011

-> 000000 01000 01001 01010 00000 100011

\begin{lstlisting}
op=0x0, rs=0x08, rt=0x09, rd=0x0A, shamt=0x0, funct=0x23
\end{lstlisting}
\begin{description}
\item[Format:] R
\item[MIPS Instruction:] 
\begin{lstlisting}
subu $t2, $t0, $t1
\end{lstlisting}
\end{description}

\paragraph{Problem 6: }
Translate the following C code to MIPS assembly code. Assume that the values of $a$, $b$ and $c$ are in registers $\$s1$, $\$s2$ and $\$s3$, respectively.
\begin{lstlisting}
if (a <= b)
	b = 4*a + c;
else
	b = 0;
\end{lstlisting}

\begin{lstlisting}
       ble  $s0, $s1, Else
       sll  $t0, $s0, 2
       add  $s1, $t0, $s3
       beq  $zero, $zero Endif
Else:  sw   $zero, 0($s1) 
Endif: ..
\end{lstlisting}

\paragraph{Problem 7: }
Translate the following C code to MIPS assembly code. Assume that the values of $i$ and $n$ are in registers $\$s0$ and $\$s1$, respectively. Also, assume that register $\$s2$ holds the base address of the array $A$.

\begin{lstlisting}
for(i=2*n; i > n; i--) {
	A[i] = A[i] + A[i-1] - i;
}
\end{lstlisting}

\begin{lstlisting}
      sll $s0, $s1, 1
For:  ble $s0, $s1, Efor
      sll $t1, $s0, 2
      add $t0, $t1, $s2
      lw  $t1, 0($t0)
      sub $t2, $t1, 4
      lw  $t2, 0($t2)
      add $t1, $t1, $t2
      sub $t1, $t1, $s0
      sw  $t1, 0($t0)
      addi $s0, -1
      beq $zero, $zero For
Efor: ..
\end{lstlisting}


\end{raggedright}
\end{document}
