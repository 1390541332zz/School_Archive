\documentclass[12pt,letterpaper,titlepage]{report}
\usepackage{fontspec}
\defaultfontfeatures{Mapping=tex-text}
\usepackage{xunicode}
\usepackage{xltxtra}
\usepackage{enumitem}
\setmainfont{Times New Roman}
\usepackage{amsmath}
\usepackage{amsfonts}
\usepackage{amssymb}
\usepackage{multicol}
\usepackage{paracol}
\usepackage{graphicx}
\graphicspath{{img/}}
\usepackage[left=1in,right=1in,top=1in,bottom=1in]{geometry}
\usepackage{setspace}


\author{Jacob Abel}
\title{%
	Essay 2: Volkswagen Scandal Revisited
	\\\large ECE2014 CRN:82708
}
\begin{document}
\maketitle
\setstretch{2} 

\textbf{\emph{My perspective on the following ethical discussion is identical to my perspective in Essay 1.}}\bigskip

Supposed that I was an engineer involved in the Volkswagen emissions falsification scandal given the following conditions. Before becoming aware of the ethical dilemma, I have become deeply involved in the project. The culture surrounding the project largely supports the initiative. I am a rank and file member without significant capacity to leave Volkswagen. With these conditions established, the following narrative will discuss my response to the situation as well as any rationale or motivations.

Starting with my initial reaction, I would likely find myself extremely uncomfortable as I strongly believe in the moral responsibility of a company to accurately and truthfully represent their products. This is with the additional stress as a result of the possibility of legal retribution for participation in the project. My most likely response would be to attempt to reduce involvement with the project. While this may only have limited success, reducing involvement atleast minimises additional potential threats to my well-being, both financially and legally.

Now that I have had some time to attempt to distance myself from the project and consider my options, I can start to try and do my part in resolving the issue. I would initially attempt to begin supplying anonymous tips to relevant authorities. This provides a safe, minimal risk channel for disclosing the issues without directly involving myself. This also provides a means of insulating myself from the fallout should I keep records of my anonymous tips. Should little to nothing come of attempting to tip off the authorities, the next most viable option would be to anonymously release incriminating documentation to the media. I would sanitise the documentation first as to avoid directly incriminating specific employees or indicating that I had been the one to leak the files. Similarly to with the anonymous tips, I could insulate myself from the fallout by digitally signing the documentation with a private key that I could field should proof of my involvement need come to light. With this, I believe that I would have done my due diligence and should nothing change, I would begin job searching and looking for other opportunities.

I would like to think that had I been in this scenario, this is the course of action I would have taken however talk is cheap and there is a significant difference between theorising about taking the high road and actually doing so. I believe that because of this out of all the possible ways to ethically handle the scenario, my theorised course of action would be the easiest to carry out. This is largely due to the fact that my course of action largely absolves myself of responsibility for disclosing the scandal while maintaining an anonymous way of proving my due diligence. 


\end{document}