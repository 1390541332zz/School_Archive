\documentclass[12pt,letterpaper,titlepage]{article}

\usepackage{fontspec}
\defaultfontfeatures{Mapping=tex-text}
\usepackage{xunicode}
\usepackage{xltxtra}
\usepackage{amsmath}
\usepackage{pdfpages}
\usepackage{amsfonts}
\usepackage{amssymb}
\setcounter{secnumdepth}{0}
\usepackage{nameref}
\usepackage{enumitem}
\usepackage{environ}

\setmainfont{Times New Roman}
\showboxdepth=\maxdimen
\showboxbreadth=\maxdimen


\usepackage{paracol}
\usepackage{wrapfig}
\globalcounter{table}
\globalcounter{figure}
\usepackage{graphicx}
\usepackage[left=1in,right=1in,top=1in,bottom=1in]{geometry}
\graphicspath{{img/}}

\author{Jacob Abel}
\title{	Homework 4
	\\\large ECE2004 CRN:12898
}

\setlength{\parskip}{0.5em}

\begin{document}
\maketitle
\begin{raggedright}

\paragraph{Question 1: }

Find the Thevenin equivalent voltage and resistance at terminals $a-b$.

\begin{center}
\includegraphics[width=\textwidth, height=\textheight, keepaspectratio=true]{hw4q1}
\end{center}


\clearpage

\paragraph{Question 2: }

Find the Norton equivalent current and resistance at terminals $a-b$.

\begin{center}
\includegraphics[width=\textwidth, height=\textheight, keepaspectratio=true]{hw4q2}
\end{center}


\clearpage

\paragraph{Question 3: }

What is the maximum power that can be delivered to $R_L$?

\begin{center}
\includegraphics[width=\textwidth, height=\textheight, keepaspectratio=true]{hw4q3}
\end{center}


\clearpage

\paragraph{Question 4: }

\begin{center}
\includegraphics[width=0.9\textwidth, height=\textheight, keepaspectratio=true]{hw4q4}
\end{center}

\begin{enumerate}[label=\Alph*)]
\item Use superposition to find $V_x$.
\item Analyse the full circuit with mesh current or nodal analysis to find $V_x$.
\end{enumerate}

\end{raggedright}
\end{document}
