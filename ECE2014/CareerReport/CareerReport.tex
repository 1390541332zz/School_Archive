\documentclass[12pt,letterpaper,titlepage]{report}
\usepackage{fontspec}
\defaultfontfeatures{Mapping=tex-text}
\usepackage{xunicode}
\usepackage{xltxtra}
\usepackage{enumitem}
\setmainfont{Times New Roman}
\usepackage{amsmath}
\usepackage{amsfonts}
\usepackage{amssymb}
\usepackage{multicol}
\usepackage{paracol}
\usepackage{graphicx}
\graphicspath{{img/}}
\usepackage[left=1in,right=1in,top=1in,bottom=1in]{geometry}
\usepackage{setspace}


\author{Jacob Abel}
\title{%
	Career Report
	\\\large ECE2014 CRN:82708
}
\begin{document}
\maketitle
\begin{raggedright}
\setlength{\parindent}{.5cm}

The ECE field that I am primarily interested in is the software systems domain. The basic binding aspect of this domain is that all the projects are built around complex systems driven primarily by reprogrammable logic of some form be it field programmable gate arrays, general purpose CPUs, or specialised graphics/compute pipelines. The primary focus on software solutions is central to one of the main appeals of the software systems domain which is that essentially any aspect of the project can be modified in very little time with minimal access to the hardware and often no need to modify the hardware. Due to this software systems are often cheaper than dedicated hardware solutions. Additionally, where hardware based products have a fairly complex unit cost due to the logistics of manufacturing, packaging, shipping, and storing of the product, purely software based products often have a per-unit cost of zero. The capability to perform revisions in rapid succession with minimal capital cost is one of the primary reasons that startups and entrepreneurs like those that preside in Silicon Valley and the San Francisco Bay area have found such rapid and massive growth.

Due to the primary focus on programming and the rapid growth in the past few decades the domain has a range of many vastly different software ecosystems, each with their own unique styles, cultures, and paradigms. Several major communities that I find of particular interest are the embedded software design, artificial intelligence, and cryptocurrency \& decentralised services communities.

The embedded software design sub-domain receives significantly less fanfare than many of the other software systems domains such as web/app front-end development and artificial intelligence despite being one of the oldest and most important sub-domains as it forms the backbone for every other major field within the field of software systems. Embedded software design is in essence the development, maintenance, and improvement of software that interfaces with the bare-metal. These projects largely consist of developing a consistent interface for higher level software to interact with peripherals and the outside world. The projects aren't all writing drivers for other developers though. There are many cases where the entire software system is implemented in a bare-metal environment due to design constraints. As such embedded software is present in essentially everything from cars to smartphones to rockets and Martian rovers. Nowadays embedded software is also present in most refrigerators, coffee pots, radios, and even some light-bulbs. Do to this it is a field with an enormous amount of demand. While the embedded field has been around longer than most other fields in the domain, there is still turbulence in the industry. One of the sources of this turbulence is the influx of new engineers and developers who were raised with many of the niceties of modern high level languages and as such struggle with adapting to the common languages of the embedded domain such as C and Assembly which require significantly more attention to detail to prevent errors but produce extremely high performance code.

The field of artificial intelligence and machine learning is often a focus of science fiction and pop culture. Despite media showing AI as a vastly intelligent entity that either leads humans to enlightenment or destruction, the reality is much less grandiose. Artificial intelligence is a fairly young field in its modern state despite its origin in the 70's due to poor funding up until recently. Now that AI has returned to the spotlight, the field has been progressing at breakneck speed and has made significant strides. Despite this artificial intelligence still has significant limitations. Most artificial intelligence systems are designed using networks trained on a data set to perform a task however due to this, they are effectively a black box that cannot be debugged easily. Similarly, AI are trained to do specific tasks well but are often incapable of handling all cases predictably. One example of this is the concept of adversarial attacks which trick the AI into mis-categorising images by applying a noise mostly imperceptible to humans. This noise tilts the weighting systems used by the AI into believing that a cat is a gun or that a stop sign is non-existent.(OpenAI) Combining this with the limited debugging capabilities of most modern AI, the field has significant progress to make before any self driving cars are widely considered safe enough to operate without a passenger, let alone sky-net coming to fruition.

The blockchain/cryptocurrency/distributed services field is one of the newest fields in the software systems domain but is showing significant potential in revolutionising the way technology operates. To provide a brief overview of the field, blockchains are tree data structures in which the hash of every child node is included in the hash of the parent node. This allows someone to verify that the contents of a tree have not been modified illegitimately as the process of reconstructing the tree gets exponentially more expensive the deeper into the tree the change is. This allows a service such as a cryptocurrency to operate on a network of independent actors who all have their own motives without having to worry about the actors supplying bad data be it accidental or malicious. By decentralising currencies, much of the hassles of dealing with currency exchange is automated and secured in such a way that the only party with access to your account by default is yourself. Additionally, new technologies such as ethereum and smart contracts introduced the concept of code as contract which proposes that a formally verified script is equivalent to a legally binding contract and as such can be used for the automated exchange of goods and services with zero possibility for malice from either party. These technologies are facilitating the development of autonomous currency exchanges which operate purely on network concensus and free market principles. Extending upon these ideas similar exchanges are in development that bridge the divide between cryptocurrencies and physical cash while maintaining 100\% autonomy and decentralised operation. Another area where these technologies are having an effect is in the cloud. Technologies such as IPFS, Golem Network, and Siacoin are providing a marketplace for cloud storage and computation with completely opaque payloads. These technologies are still largely developing but show great promise. 

As for why specifically I am interested in software systems, I find much of the technology interesting however I plan on starting a company after university and have found software systems to be a very bountiful field when it comes to opportunities for businesses. Additionally the low capital cost for entry into the market is helpful in minimising risk. Overall however there are just so many unique opportunities to learn different things that all operate on the same underlying principles and I find beauty in it all. My biggest issue with this domain is being forced to decide where to focus my attention as I only wish that I could have the time and resources to learn it all.







\end{raggedright}
\end{document}