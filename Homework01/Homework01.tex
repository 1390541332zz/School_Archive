\documentclass[12pt,letterpaper,notitlepage]{report}
\usepackage{fontspec}
\defaultfontfeatures{Mapping=tex-text}
\usepackage{xunicode}
\usepackage{xltxtra}
\usepackage{enumitem}
\setmainfont{Times New Roman}
\usepackage{amsmath}
\usepackage{amsfonts}
\usepackage{amssymb}
\usepackage[margin=0.65in]{geometry}
\author{Jacob Abel}
\title{%
	Homework 1
	\\\large ECE2504 CRN:82729
}

\begin{document}
\maketitle
\paragraph{Question 1}
Convert the following binary numbers to decimal. (Put the integer portion of your answer before the radix point, and the fractional portion after, as you usually do.)
\begin{enumerate}[label=\alph*)]
\item $1101100.01    = 64+32+8+4+0.25 = 108.25$
\item $1011101.101   = 64+16+8+4+1+0.5+0.125 = 93.625$
\item $10110010.1011 = 128+32+16+2+0.5+0.125+0.0625 = 178.6875$
\end{enumerate}

\paragraph{Question 2}
Convert the following numbers with the indicated bases to decimal. 
\begin{enumerate}[label=\alph*)]
\item $(20121)_{3} = (81 \times 2) + (9 \times 1) + (3 \times 2) + (1 \times 1) = 178_{10}$
\item $(13201)_{6} = (1296 \times 1) + (216 \times 3) + (36 \times 2) + (1 \times 1) = 2017_{10}$
\item $(1492)_{11} = (1331 \times 1) + (121 \times 4) + (11 \times 9) + (1 \times 2) = 1916_{10}$
\end{enumerate}

\paragraph{Question 3} 
Convert the following decimal numbers to binary. (Put the integer portion of your answer before the radix point, and the fractional portion after, as you usually do.) 
\begin{enumerate}[label=\alph*)]
\item $493.6015625_{10} = 2^8+2^7+2^6+2^5+2^3+2^2+2^0+2^{-1}+2^{-4}+2^{-5}+2^{-7} = 0b111101101.1001101$
\item $1989.28515625_{10} = 2^{10}+2^9+2^8+2^7+2^6+2^2+2^0+2^{-2}+2^{-5}+2^{-8}= 0b11111000101.01001001$
\item $10052.58203125_{10} = 2^{13}+2^{10}+2^9+2^8+2^6+2^2+2^{-1}+2^{-4}+2^{-6}+2^{-8} = 0b10011101000100.10010101$
\end{enumerate}

\paragraph{Question 4} 
Convert the decimal numbers given in Problem 3 to hex. (Put the integer portion of your answer before the radix point, and the fractional portion after, as you usually do.)
\begin{enumerate}[label=\alph*)]
\item $493.6015625_{10} = (16^2\times1)+(16^1\times14)+(16^0\times13)+(16^{-1}\times9)+(16^{-2}\times10) = 0x2ED.9A$
\item $1989.28515625_{10} = (16^2\times7)+(16^1\times12)+(16^0\times5)+(16^{-1}\times4)+(16^{-2}\times9) = 0x7C5.49$
\item $10052.58203125_{10} = (16^3\times2)+(16^2\times7)+(16^1\times4)+(16^0\times4)+(16^{-1}\times9)+(16^{-2}\times5) = 0x2744.95$
\end{enumerate}

\paragraph{Question 5} 
Complete the table by converting the numbers from the given base to the other three bases. (Put the integer portion of your answer before the radix point, and the fractional portion after, as you usually do. Depending on the width of your screen, it's possible that the fields for the integer and fraction portions may show up on separate lines.)\medskip
\begin{center}
\def\arraystretch{1.5}
\begin{tabular}{ | c | c | c | c | } \hline
Decimal      & Binary              & Octal    & Hex    \\\hline
523.60546875 & 1000001011.10011011 & 1013.466 & 20B.9B \\\hline
173.4375     & 10101101.0111       & 255.34   & AD.7   \\\hline
469.46875    & 111010101.011110    & 725.36   & 1D5.78 \\\hline
2771.5       & 101011010011.1      & 5323.4   & AD3.8  \\\hline
\end{tabular}
\end{center}
To Binary
\\$ 523.60546875 = 2^9+2^3+2^1+2^0+2^{-1}+2^{-4}+2^{-5}+2^{-7}+2^{-8} = 0b1000001011.10011011 $
\\$ 725.36_8 = (7)111\; (2)010\; (5)101\; .\; (3)011\; (6)110 = 0b111010101.011110$
\\$ 0xAD3.8 = (A)1010\; (D)1101\; (3)0011 \;.\; (8)1000 = 0b101011010011.1$
\\To Hex
\\$ 523.60546875 = (0010)2\; (0000)0\; (1011)B\; .\; (1001)9\; (1011)B = 0x20B.9B $
\\$ 0b10101101.0111 = (1010)A\; (1101)D \;.\; (0111)7 = 0xAD.7$
\\$ 725.36_8 = (0001)1 \;(1101)D \;(0101)5\; . \;(0111)7 \;(1000)8 = 0x1D5.78$
\\To Octal
\\$ 523.60546875 = (001)1\; (000)0\; (001)1\; (011)3 \;.\; (100)4 (110)6 (110)6 = 1013.466_8 $
\\$ 0b10101101.0111 = (010)2 \;(101)5 \;(101)5 \;.\; (011)3 \;(100)4 = 255.34_8$
\\$ 0xAD3.8 = (5)101\; (3)011\; (2)010\; (3)011 \;. \; (4)100 = 5323.4_8$
\\To Decimal
\\$ 0b10101101.0111 = 2^7+2^5+2^3+2^2 + 2^0 + 2^{-2} + 2^{-3} + 2^{-4} = 173.4375_{10}$
\\$ 725.36_8 = (8^2\times7)+(8^1\times2)+(8^0\times5)+(8^{-1}\times3)+(8^{-2}\times6) = 469.46875_{10}$
\\$ 0xAD3.8 = (16^2\times10)+(16^1\times13)+(16^0\times3)+(16^{-1}\times8) = 2771.5_{10} $

\paragraph{Question 6} 
What is the decimal equivalent of the largest (unsigned) binary integer that can be obtained with
\begin{enumerate}[label=\alph*)]
\item 12 bits? $ 2^{12}-1 = 4095 $
\item 22 bits? $ 2^{22}-1 = 4194303$
\end{enumerate}

\paragraph{Question 7} 
A number system uses base 12 (duodecimal). There are at most four integer digits. The weights of the digits are 123, 122, 12, and 1. Special names are given to the weights as follows:\\
$12$ = 1 dozen\\
$12^2$ = 1 gross\\
$12^3$ = 1 great gross
\begin{enumerate}[label=\alph*)]
\item How many beverage cans are in 6 great gross, 11 gross, 7 dozen and 5? \\
$ (12^3\times6)+(12^2\times11)+(12^1\times7)+(12^0\times5) = 12041 $ (in decimal)
\item Find the representation in base 12 for $7496_{10}$ beverage cans. \\
$(12^3\times4)+(12^2\times4)+(12^0\times8) = 7496$\\ 
4 great gross, 4 gross, 0 dozen, 8 cans
\end{enumerate}

\end{document}