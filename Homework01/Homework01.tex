\documentclass[12pt,letterpaper,titlepage]{article}

\usepackage{fontspec}
\defaultfontfeatures{Mapping=tex-text}
\usepackage{xunicode}
\usepackage{xltxtra}
\usepackage{amsmath}
\usepackage{pdfpages}
\usepackage{amsfonts}
\usepackage{amssymb}
\setcounter{secnumdepth}{0}
\usepackage{nameref}
\usepackage{enumitem}

\setmainfont{Times New Roman}
\showboxdepth=\maxdimen
\showboxbreadth=\maxdimen

\usepackage[tocflat]{tocstyle}
\usetocstyle{allwithdot}
\usepackage[bottom]{footmisc}

\usepackage{paracol}
\usepackage{wrapfig}
\globalcounter{table}
\globalcounter{figure}
\usepackage{graphicx}
\usepackage[left=1in,right=1in,top=1in,bottom=1in]{geometry}
\graphicspath{{img/}}

\author{Jacob Abel}
\title{	Homework 1
	\\\large ECE2004 CRN:12898
}

\setlength{\parskip}{0.5em}

\begin{document}
\maketitle

\paragraph{Question 1: }

\begin{center}
\includegraphics[width=.5\textwidth, height=\textheight, keepaspectratio=true]{hw1q1}
\end{center}

\begin{enumerate}[label=\Alph*.]
\item The power absorbed by this element is -100 mW. Find $R_x$ and $i_x$.
\begin{align*}
i_x =& \frac{P}{V} = \frac{-100mW}{10V} = -0.01A        \\
R_x =& \frac{V^2}{P} = \frac{(10V)^2}{-100mW} = -1000 \Omega
\end{align*}
\item Which direction is the current $i_x$ flowing? From negative to positive.
\item Is this element producing or consuming power? It is producing power.
\end{enumerate}

\clearpage

\paragraph{Question 2: }

KCL states current entering a node is equal to current leaving a node. There is, however, no reason to restrict ourselves to simply nodes. It may be restated as currents entering a closed surface have to equal currents leaving a closed surface. That being said, solve for the currents ($I_1$,$I_2$,$I_3$).

\begin{center}
\includegraphics[width=.75\textwidth, height=\textheight, keepaspectratio=true]{hw1q2}
\end{center}

\begin{align*}
  \text{Closed Surface 1} 	=& -40mA + 30mA + 100mA + 10mA - I_2
\\							=& 0.1A - I_2
\\						I_2 =& 0.1A
\\\text{Closed Surface 2} 	=& -10mA - 100mA - I_1
\\						I_1 =& -0.11A
\\I_1I_2I_3 \text{Node} 	=& I_1 + I_2 + I_3
\\ 						I_3 =& -I_1 - I_2
\\ 						I_3 =& 0.11A - 0.1A
\\						I_3 =& 0.01A
\\
\\		    (I_1, I_2, I_3) =& (-0.11A, 0.1A, 0.01A)
\end{align*}

\clearpage

\paragraph{Question 3: }

\begin{center}
\includegraphics[width=\textwidth, height=\textheight, keepaspectratio=true]{hw1q3}
\end{center}

\begin{enumerate}[label=\Alph*.]
\item Find the current $i$.

\begin{align*}
   -5V + v_{1\Omega} + v_{9\Omega} + 10V + v_{10\Omega} 	&= 0
\\ v_{1\Omega} + v_{9\Omega} + v_{10\Omega} 				&= -5V
\\ 1\Omega \times i + 9\Omega \times i + 10\Omega \times i 	&= -5V
\\ 20\Omega \times i 										&= -5V
\\ i 														&= \frac{-5V}{20\Omega}
\\ i 														&= -0.25A
\end{align*}

\item Find $V_{AB}$, the voltage across the $1 \Omega$ resistor. 
$V_{AB} = 1\Omega \times -0.25A = -0.25V$
\item Is the voltage at node B higher or lower than the voltage at node A?
The voltage is higher at node A than at node B as the potential difference $V_{AB}$ is negative indicating that the voltage is decreasing as you approach B from A.
\end{enumerate}



\clearpage

\paragraph{Question 4: }

\begin{center}
\includegraphics[width=.75\textwidth, height=\textheight, keepaspectratio=true]{hw1q4}
\end{center}

\begin{enumerate}[label=\Alph*.]
\item Solve for the current ($i_1$, $i_2$, $i_3$) as well as $V_x$.

\begin{align*}
   -30V + i_1 \times (5\Omega + 3\Omega + \frac{1}{\frac{1}{6\Omega}+\frac{1}{3\Omega}}) &= 0
\\ i_1 \times (5\Omega + 3\Omega + 2\Omega) 		&= 30V
\\ i_1												&= \frac{30V}{10\Omega}
\\ i_1												&= 3A 
\\ i_2												&= 3A \frac{2\Omega}{6\Omega}
\\ i_2												&= 1A 
\\ i_3												&= 3A \frac{2\Omega}{3\Omega}
\\ i_3												&= 2A
\\ i_1 - i_2 - i_3									&= 0A
\\ (i_1, i_2, i_3)									&= (3A, 1A, 2A)
\end{align*}

\item What is the voltage across the $5 \Omega$ resistor ($V_{AB}$)?
$V_{AB} = 5\Omega \times 3A = 15V$
 
\end{enumerate}


\begin{raggedright}



\end{raggedright}
\end{document}
