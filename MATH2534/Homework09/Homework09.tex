\documentclass[12pt,letterpaper,titlepage]{article}

\usepackage{fontspec}
\defaultfontfeatures{Mapping=tex-text}
\usepackage{xunicode}
\usepackage{xltxtra}
\usepackage{amsmath}
\usepackage{pdfpages}
\usepackage{mathtools}
\usepackage{amsfonts}
\usepackage{amssymb}
\setcounter{secnumdepth}{0}
\usepackage{nameref}
\usepackage{enumitem}
\usepackage{environ}

\showboxdepth=\maxdimen
\showboxbreadth=\maxdimen

\usepackage[tocflat]{tocstyle}
\usetocstyle{allwithdot}
\usepackage[bottom]{footmisc}

\usepackage{paracol}
\usepackage{wrapfig}
\globalcounter{table}
\globalcounter{figure}
\usepackage{graphicx}
\usepackage[left=1in,right=1in,top=1in,bottom=1in]{geometry}
\graphicspath{{img/}}

\author{Jacob Abel}
\title{	Homework 9
	\\\large MATH2534 CRN:15708
}

\setlength{\parskip}{0.5em}

\begin{document}
\maketitle
\begin{raggedright}

\begin{enumerate}

\item Prove by set identities (or by element method) or disprove the following:

\begin{enumerate}[label=(\alph*)]
\item If $B\subseteq C$, then $(A-B)\subseteq (A-C)$

$A=\{1, 2, 3, 4\}, B=\{2, 3\}, C=\{2,3,4\}$

$A-B = \{1, 4\}$

$A-C = \{1\}$

$\{1, 4\} \not\subseteq \{1\}$

\item $(A\cup B)-B = A$

$A=\{1, 2, 3, 4\}, B=\{4, 5, 6\}$

$A\cup B = \{1, 2, 3, 4, 5, 6\}$

$(A\cup B) - B = \{1, 2, 3\}$

$(A\cup B) - B \neq A$

\item $A\cup(B-C)=(A\cup B)-(A\cup C)$

$A=\{1, 2, 3\}, B=\{2, 3, 4\}, C=\{2,3\}$

$B-C = \{4\}$

$A\cup(B-C) = \{1, 2, 3, 4\}$

$A\cup B = \{1, 2, 3, 4\}$

$A\cup C = \{1, 2, 3\}$

$(A\cup B)-(A\cup C) = \{4\}$

$A\cup(B-C) \neq (A\cup B)-(A\cup C)$

\end{enumerate}

\item Simplify $(A\cap(B\cup C))\cap(A-B)\cap(B\cup C^\mathcal{C})$

$(A\cap(B\cup C))\cap(A-B)\cap(B\cup C^\mathcal{C}) = A\cap(B\cup C)\cap A\cap B^\mathcal{C}\cap (B\cup C^\mathcal{C})$

By set difference definition and associative laws

$A\cap(B\cup C)\cap A\cap B^\mathcal{C}\cap (B\cup C^\mathcal{C}) = (B\cup C)\cap A\cap B^\mathcal{C}\cap (B\cup C^\mathcal{C})$

By idempotent laws

$(B\cup C)\cap A\cap B^\mathcal{C}\cap (B\cup C^\mathcal{C}) = A\cap B^\mathcal{C}\cap (B\cup(C\cap C^\mathcal{C}))$

by distributive laws

$A\cap B^\mathcal{C}\cap (B\cup(C\cap C^\mathcal{C})) = A\cap B^\mathcal{C}\cap B$

by complement laws and universal bounds laws

$A\cap B^\mathcal{C}\cap B = A\cap \emptyset$

by complement laws

$A\cap \emptyset = \emptyset$

by universal bounds laws

 $(A\cap(B\cup C))\cap(A-B)\cap(B\cup C^\mathcal{C}) = \emptyset$

\clearpage

\item Prove that for all sets $A$ and $B$, $\mathcal{P}(A)\cap \mathcal{P}(B) = \mathcal{P}(A\cap B)$.

Suppose that $\forall x$ if $x\in \mathcal{P}(A\cap B)$, then $x\in \mathcal{P}(A)\cap \mathcal{P}(B)$

$x \subseteq A\cap B$ by definition of a power set

$x \subseteq A, x\subseteq B$ by definition of an intersection

$x\in \mathcal{P}(A), x\in\mathcal{P}(B)$ by definition of a power set

$x\in \mathcal{P}(A)\cap \mathcal{P}(B)$ by definition of an interesection

Suppose that $\forall x$ if $\mathcal{P}(A)\cap \mathcal{P}(B)$, then $x\in \mathcal{P}(A\cap B)$

$x\in \mathcal{P}(A), x\in\mathcal{P}(B)$ by definition of an intersection

$x \subseteq A, x\subseteq B$ by definition of a power set

$x \subseteq A\cap B$ by definition of an intersection

$x\in \mathcal{P}(A\cap B)$ by definition of a power set

Therefore as $x\in \mathcal{P}(A\cap B)$ and $x\in \mathcal{P}(A)\cap \mathcal{P}(B)$, $\mathcal{P}(A)\cap \mathcal{P}(B) = \mathcal{P}(A\cap B)$.


\item Let $A$, $B$, and $C$ be sets. Prove that $(A-B)-(B-C)=A-B$ with double set inclusion.

Let $\exists x \in U$

Suppose that $\forall x$ if $x\in (A-B)-(B-C)$, then $x \in A - B$.

$x\in A, x\not\in B, x\not in (B - C)$ by set difference definition

$x\in A, x\not\in B$ by absorption rule

$x\in (A - B)$ by set difference definition

Suppose that $\forall x$ if $x \in A - B$, then $x\in (A-B)-(B-C)$.

$x\in A, x\not\in B$ by set difference definition

$x\in A, x\not\in B, x\not\in (B - C)$ by absorption rule

$x\in (A - B), x\not\in (B-C)$ by set difference definition

$x\in (A - B) - (B-C)$ by set difference definition

Therefore as $x\in (A - B)$ and $x\in (A - B) - (B-C)$, $(A-B)-(B-C)=A-B$.

\clearpage

\item Consider the set $A$ of all divisors of 70, $A = \{1, 2, 5, 7, 10, 14, 35, 70\}$, with the operations defined as follows: $a + b = \text{GCD}(a, b)$, $a \cdot b = \text{LCM}(a, b)$ and the complement is defined to be $a = 70 / a$. Assume that $A$ is a Boolean algebra with operations $+$ and $\cdot$.

\begin{enumerate}[label=(\alph*)]
\item Evaluate $14\cdot 35$, $5+\overline{5}$, and $14\cdot\overline{14}$

$14\cdot 35 = \text{LCM}(14, 35) = 70$

$5+\overline{5} = 5 + 70 / 5 = 5 + 14 = \text{GCD}(5, 14) = 1$

$14\cdot\overline{14} = 14\cdot 70/14 = 14\cdot 5  = \text{LCM}(14, 5) = 70$

\item Show $2+(7\cdot 14)=(2+7)\cdot(2+14)$

$2+(7\cdot 14)=2+\text{LCM}(7, 14)= 2 + 14 = \text{GCD}(2, 14) = 2$

$(2+7)\cdot(2+14) = \text{GCD}(2, 7)\cdot\text{GCD}(2, 14) = \text{LCM}(1, 2) = 2$

$2 = 2$

$2+(7\cdot 14)=(2+7)\cdot(2+14)$

\item Show $\overline{10+35}=\overline{10}\cdot\overline{35}$

$\overline{10+35} = 70 / \text{GCD}(10, 35) = 70 / 5 = 14$

$\overline{10}\cdot\overline{35} = 70 / 10 \cdot 70 / 35 = 7 \cdot 2 = \text{LCM}(7, 2) = 14$

$14 = 14$

$\overline{10+35}=\overline{10}\cdot\overline{35}$

\item Find the identities 0 and 1 for operations $+$ and $\cdot$, respectively. Justify your answer.

$a + 0 = \text{GCD}(a, 0) = a$ 0 is divisible by all integers

$a + 1 = \text{GCD}(a, 1) = 1$ 1 is only divisible by 1

$a \cdot 0 = \text{LCM}(a, 0) = 0$ 0 multiplied by anything is 0

$a \cdot 1 = \text{LCM}(a, 1) = a$ The GCD of any number, $a$ and one is $a$.

\item Verify $a+\overline{a}=1$ and $a\cdot\overline{a}=0$ for all $a$ in $A$.

$a + \overline{a} = a + 70 / a = \text{GCD}(a, 70 / a) = 1$ $70 / a$ will never be divisible by $a$ and either a or $\overline{a}$ will be prime.

$a\cdot\overline{a}= \text{LCM}(a, 70 / a) = 70$ The LCM of $a$ and $70 / a$ will always be 70 as the $\frac{1}{a}$ must always be canceled out first.

\end{enumerate}

\item Assume that $B$ is a Boolean algebra with operations $+$ and $\cdot$. For all $a$ and $b$ in $B$, prove the absorption law $(a + b) \cdot a = a$.

$(a + b) \cdot a = (a+T)\cdot(a+b)$ by identity law

$(a+T)\cdot(a+b) = a \cdot (1+b)$ by distributive law

$a \cdot (1+b) = a \cdot 1$ by universal bounds law

$a \cdot 1 = a$ by identity law

$(a + b) \cdot a = a$

\end{enumerate}
\end{raggedright}
\end{document}
