\documentclass[12pt,letterpaper,titlepage]{article}

\usepackage{fontspec}
\defaultfontfeatures{Mapping=tex-text}
\usepackage{xunicode}
\usepackage{xltxtra}
\usepackage{amsmath}
\usepackage{pdfpages}
\usepackage{mathtools}
\usepackage{amsfonts}
\usepackage{amssymb}
\setcounter{secnumdepth}{0}
\usepackage{nameref}
\usepackage{enumitem}
\usepackage{environ}

\showboxdepth=\maxdimen
\showboxbreadth=\maxdimen

\usepackage[tocflat]{tocstyle}
\usetocstyle{allwithdot}
\usepackage[bottom]{footmisc}

\usepackage{paracol}
\usepackage{wrapfig}
\globalcounter{table}
\globalcounter{figure}
\usepackage{graphicx}
\usepackage[left=1in,right=1in,top=1in,bottom=1in]{geometry}
\graphicspath{{img/}}

\author{Jacob Abel}
\title{	Homework 4
	\\\large MATH2534 CRN:15708
}

\setlength{\parskip}{0.5em}

\begin{document}
\maketitle
\begin{raggedright}
Prove the following using definitions given in class.
\begin{enumerate}

%==============================================================%

\item The square of any rational number subtracted from 5 is a rational number.

Let $\forall n\in \mathbb{Q}, \exists a, b \in \mathbb{Z}, b \neq 0$.

As $n$ is rational it is the quotient of some $a$ and $b$.

\begin{equation}
n  = \frac{a}{b}
\end{equation}

$n^2$ is a composite of $n$ and $n$ and therefore is a composite of $\frac{a}{b}$ and $\frac{a}{b}$

\begin{equation}
n^2 = n \times n = \frac{a}{b} \times \frac{a}{b}
\end{equation}

By simplifying, it can be shown that $n^2$ is still a rational number.

\begin{equation}
n^2 = \frac{a}{b} \times \frac{a}{b} = \frac{a^2}{b^2}
\end{equation}

As 5 is an integer and therefore a rational number and $\frac{b^2}{b^2}$ is equivalent to 1, $5\times \frac{b^2}{b^2}$ is still rational. It is now shown that the fully simplified form is solely composed of a sum of integers divided by integers. 
\begin{equation}
5 - n^2 = 5 - \frac{a^2}{b^2} 
        = \frac{b^2}{b^2} \times 5 - \frac{a^2}{b^2} 
        = \frac{5b^2-a^2}{b^2}
\end{equation}

Therefore by the definition of a rational number, $n^2-5$ is rational.

\pagebreak

%==============================================================%

\item If $x$ is any nonzero rational number and $y$ is any rational number, then $\frac{5y}{x}$ is rational.

Let $\forall x, y \in\mathbb{Q}, 
	 \exists a,b,c,d \in \mathbb{Z}, 
	 x \neq 0, 
	 a \neq 0, 
	 b \neq 0, 
	 d \neq 0$

As $x$ and $y$ are rational, they are equivalent to $\frac{a}{b}$ and $\frac{c}{d}$ respectively where $a$, $b$, and $d$ are not 0.
\begin{equation}
x = \frac{a}{b}, y = \frac{c}{d}
\end{equation}

As $y\times 5$ is equivalent to $y$ added to itself 5 times, $5y$ is still rational.
\begin{equation}
5y = y + y + y + y + y
\end{equation}

As $5bc$ and $ad$ are composites of integers, they are still integers. Additionally $ad$ is non zero due to the zero product property.

\begin{equation}
\frac{5y}{x} = \frac{5\frac{c}{d}}{\frac{a}{b}} = \frac{5bc}{ad}
\end{equation}

Therefore $\frac{5y}{x}$ is rational.

\pagebreak

%==============================================================%

\item For all integers $a$, $b$, and $c$, if $a | b$ and $a | c$, then $a | (5b − 7c)$.

Let $\forall a, b, c \in \mathbb{Z}, \exists x, y \in \mathbb{Z}$

As $a | b$, $b$ is a composite of $a$ and some other integer $x$. Similarly $a | c$, $c$ is a composite of $a$ and some other integer $y$.

\begin{equation}
5b - 7c = 5ax - 7ay
\end{equation}

By factoring the expression, we can show that $5b-7c$ is a composite of $a$ and $5x-7y$.

\begin{equation}
5b - 7c = 5ax - 7ay = (a)(5x-7y)
\end{equation}

As $5b-7c$ is a composite containing $a$, it is divisible by $a$.

\begin{equation}
\frac{5b-7c}{a} = \frac{(a)(5x-7y)}{a} = 5x-7y
\end{equation}

Therefore $a|(5b-7c)$.






\end{enumerate}
\end{raggedright}
\end{document}
