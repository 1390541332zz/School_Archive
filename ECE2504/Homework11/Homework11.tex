\documentclass[12pt,letterpaper,titlepage]{report}
\usepackage{fontspec}
\defaultfontfeatures{Mapping=tex-text}
\usepackage{xunicode}
\usepackage{xltxtra}
\usepackage{enumitem}
\setmainfont{Times New Roman}
\usepackage{amsmath}
\usepackage{amsfonts}
\usepackage{amssymb}
\usepackage{multicol}
\usepackage{paracol}
\usepackage{multirow}
\usepackage{tikz}
\usepackage{float}
\usepackage{graphicx}
\usepackage{subcaption}
\usepackage{tikz-timing}
\usetikzlibrary{automata,positioning}
\graphicspath{{img/}}
\usepackage{karnaugh-map}
\usepackage[margin=0.65in]{geometry}
\usepackage{xparse}
\author{Jacob Abel}
\title{%
	Homework 11
	\\\large ECE2504 CRN:82729
}



\begin{document}
\maketitle
\begin{raggedright}
\raggedcolumns

\paragraph{Question 1:}
(2 pts) How many flip flop values are complemented in an 6‐bit binary ripple counter to reach the next count value after

\begin{enumerate} [noitemsep, label=\alph*)]
\item 111111 : 6 Flip flops
\item 010101 : 2 Flip flops
\end{enumerate}

\paragraph{Question 2:}
(3 pts) Use D type flip flops and gates to design a counter with the following repeated sequence: 00, 01, 10

\begin{figure}[ht]
\centering
\begin{subfigure}[b]{0.3\textwidth}
  \centering
  \begin{karnaugh-map}[2][2][1][$Q_0$][$Q_1$]
    \minterms{0}
    \indeterminants{3}
    \autoterms[0]
    \implicant{0}{0}
  \end{karnaugh-map}
  \caption{$D_0$ Karnaugh Map}
\end{subfigure}
\begin{subfigure}[b]{0.3\textwidth}
  \centering
  \begin{karnaugh-map}[2][2][1][$Q_0$][$Q_1$]
     \minterms{1}
    \indeterminants{3}
    \autoterms[0]
    \implicant{1}{3}
  \end{karnaugh-map}
  \caption{$D_1$ Karnaugh Map}
\end{subfigure}
\begin{subfigure}[b]{0.3\textwidth}
  \centering
  \begin{align*}
	D_0&=Q_1'Q_0'\\
	D_1&=Q_0
  \end{align*}
  \caption{Required Combination Logic}
\end{subfigure}
\begin{subfigure}[b]{\textwidth}
  \centering
  \includegraphics[width=\textwidth, height=\textheight, keepaspectratio=true]{hw11p2}
  \caption{Circuit Diagram}
\end{subfigure}

\caption{Question 2 Counter Design}
\end{figure}
  
  
\clearpage
  
\paragraph{Question 3:}
(6 pts) Use SR type flip flops and gates to design a counter with each of the following repeated sequences:

\begin{enumerate} [noitemsep, label=\alph*)]
\item 000, 001, 010, 011, 100, 101, 110, 111


\begin{figure}[H]
\centering
\begin{subfigure}[b]{0.6\textwidth}
  \centering
  \begin{tabular}{|c|c|c|c|c|}\hline 
	Current State & Next State & $SR_2$ & $SR_1$ & $SR_0$ \\ \hline 
	000 & 001 & 00 & 00 & 10 \\ \hline 
	001 & 010 & 00 & 10 & 01 \\ \hline 
	010 & 011 & 00 & 00 & 10 \\ \hline 
	011 & 100 & 10 & 01 & 01 \\ \hline 
	100 & 101 & 00 & 00 & 10 \\ \hline 
	101 & 110 & 00 & 10 & 01 \\ \hline 	
	110 & 111 & 00 & 00 & 10 \\ \hline 
	111 & 000 & 01 & 01 & 01 \\ \hline 
  \end{tabular} 
  \caption{State Table}
\end{subfigure}
\begin{subfigure}[b]{0.3\textwidth}
  \centering
  \begin{align*}
	S_0&=Q_0'\\
	R_0&=Q_0\\
	S_1&=Q_1'Q_0\\
	R_1&=Q_1Q_0\\
	S_2&=Q_2'Q_1Q_0\\
	R_2&=Q_2Q_1Q_0\\
  \end{align*}
  \caption{Required Combination Logic}
\end{subfigure}
\begin{subfigure}[b]{0.3\textwidth}
  \centering
  \begin{karnaugh-map}[2][4][1][$Q_0$][$Q_2Q_1$]
    \minterms{0,2,4,6}
    \autoterms[0]
    \implicant{0}{4}
  \end{karnaugh-map}
  \caption{$S_0$ Karnaugh Map}
\end{subfigure}
\begin{subfigure}[b]{0.3\textwidth}
  \centering
  \begin{karnaugh-map}[2][4][1][$Q_0$][$Q_2Q_1$]
    \minterms{1,5}
    \autoterms[0]
    \implicantedge{1}{1}{5}{5}
  \end{karnaugh-map}
  \caption{$S_1$ Karnaugh Map}
\end{subfigure}
\begin{subfigure}[b]{0.3\textwidth}
  \centering
  \begin{karnaugh-map}[2][4][1][$Q_0$][$Q_2Q_1$]
    \minterms{3}
    \autoterms[0]
    \implicant{3}{3}
  \end{karnaugh-map}
  \caption{$S_2$ Karnaugh Map}
\end{subfigure}
\begin{subfigure}[b]{0.3\textwidth}
  \centering
  \begin{karnaugh-map}[2][4][1][$Q_0$][$Q_2Q_1$]
    \minterms{1,3,5,7}
    \autoterms[0]
    \implicant{1}{5}
  \end{karnaugh-map}
  \caption{$R_0$ Karnaugh Map}
\end{subfigure}
\begin{subfigure}[b]{0.3\textwidth}
  \centering
  \begin{karnaugh-map}[2][4][1][$Q_0$][$Q_2Q_1$]
    \minterms{3,7}
    \autoterms[0]
    \implicant{3}{7}
  \end{karnaugh-map}
  \caption{$R_1$ Karnaugh Map}
\end{subfigure}
\begin{subfigure}[b]{0.3\textwidth}
  \centering
  \begin{karnaugh-map}[2][4][1][$Q_0$][$Q_2Q_1$]
    \minterms{7}
    \autoterms[0]
    \implicant{7}{7}
  \end{karnaugh-map}
  \caption{$R_2$ Karnaugh Map}
\end{subfigure}
\caption{Question 3a Counter Design}
\end{figure}
\pagebreak

\begin{figure}[ht]
  \centering
  \includegraphics[width=\textwidth, height=\textheight, keepaspectratio=true]{hw11p3a}
  \caption{Question 3a Circuit Diagram}
\end{figure}

\pagebreak

\item 000, 001, 010, 011, 100

\begin{figure}[H]
\centering
\begin{subfigure}[b]{0.6\textwidth}
  \centering
  \begin{tabular}{|c|c|c|c|c|}\hline 
	Current State & Next State & $SR_2$ & $SR_1$ & $SR_0$ \\ \hline 
	000 & 001 & 00 & 00 & 10 \\ \hline 
	001 & 010 & 00 & 10 & 01 \\ \hline 
	010 & 011 & 00 & 00 & 10 \\ \hline 
	011 & 100 & 10 & 01 & 01 \\ \hline 
	100 & 000 & 01 & 00 & 10 \\ \hline 
  \end{tabular} 
  \caption{State Table}
\end{subfigure}
\begin{subfigure}[b]{0.3\textwidth}
  \centering
  \begin{align*}
	S_0&=Q_2'Q_0'+Q_1'Q_0'\\
	R_0&=Q_2'Q_0\\
	S_1&=Q_2'Q_1'Q_0\\
	R_1&=Q_2'Q_1Q_0\\
	S_2&=Q_2'Q_1Q_0\\
	R_2&=Q_2Q_1'Q_0'\\
  \end{align*}
  \caption{Required Combination Logic}
\end{subfigure}
\begin{subfigure}[b]{0.3\textwidth}
  \centering
  \begin{karnaugh-map}[2][4][1][$Q_0$][$Q_2Q_1$]
    \minterms{0,2,4}
    \autoterms[0]
    \implicantedge{0}{0}{4}{4}
    \implicant{0}{2}
  \end{karnaugh-map}
  \caption{$S_0$ Karnaugh Map}
\end{subfigure}
\begin{subfigure}[b]{0.3\textwidth}
  \centering
  \begin{karnaugh-map}[2][4][1][$Q_0$][$Q_2Q_1$]
    \minterms{1}
    \autoterms[0]
    \implicant{1}{1}
  \end{karnaugh-map}
  \caption{$S_1$ Karnaugh Map}
\end{subfigure}
\begin{subfigure}[b]{0.3\textwidth}
  \centering
  \begin{karnaugh-map}[2][4][1][$Q_0$][$Q_2Q_1$]
    \minterms{3}
    \autoterms[0]
    \implicant{3}{3}
  \end{karnaugh-map}
  \caption{$S_2$ Karnaugh Map}
\end{subfigure}
\begin{subfigure}[b]{0.3\textwidth}
  \centering
  \begin{karnaugh-map}[2][4][1][$Q_0$][$Q_2Q_1$]
    \minterms{1,3}
    \autoterms[0]
    \implicant{1}{3}
  \end{karnaugh-map}
  \caption{$R_0$ Karnaugh Map}
\end{subfigure}
\begin{subfigure}[b]{0.3\textwidth}
  \centering
  \begin{karnaugh-map}[2][4][1][$Q_0$][$Q_2Q_1$]
    \minterms{3}
    \autoterms[0]
    \implicant{3}{3}
  \end{karnaugh-map}
  \caption{$R_1$ Karnaugh Map}
\end{subfigure}
\begin{subfigure}[b]{0.3\textwidth}
  \centering
  \begin{karnaugh-map}[2][4][1][$Q_0$][$Q_2Q_1$]
    \minterms{4}
    \autoterms[0]
    \implicant{4}{4}
  \end{karnaugh-map}
  \caption{$R_2$ Karnaugh Map}
\end{subfigure}
\caption{Question 3b Counter Design}
\end{figure}
\pagebreak

\begin{figure}[ht]
  \centering
  \includegraphics[width=\textwidth, height=\textheight, keepaspectratio=true]{hw11p3b}
  \caption{Question 3b Circuit Diagram}
\end{figure}

\end{enumerate}
\clearpage

\paragraph{Question 4:}
(5 pts) Draw the state diagram for the following electronic vending machine specification. The vending machine sells Squirrel Nut Zippers for 25\textcent. The machine accepts N (nickels = 5\textcent), D (dimes = 10\textcent), and Q (quarters = 25\textcent). When the sum of the coins inserted in sequence is 25\textcent or more, the machine dispenses one Squirrel Nut Zipper by making V = 1 and returns to its initial state. If less than 25\textcent is inserted and the coin return button CR is pressed, then the coins deposited are returned by making C = 1. Change is not returned if more than 25\textcent is deposited.

\paragraph{Question 5:}
A communication link requires a circuit that produces the sequence 01111110. Design a synchronous sequential circuit that starts producing this sequence when the input E=1. Once the sequence starts, it completes. If E=1 during the last output in the sequence, the sequence repeats. Otherwise (if E=0) the output remains constant at 1.

\begin{enumerate} [noitemsep, label=\alph*)]
\item (3 pts) Draw the state diagram.
\item (4 pts) Draw the state table and make a state assignment.
\item (3 pts) Design the circuit using SR flip flops and logic gates.
\end{enumerate}

\paragraph{Question 6:}
(5 pts) A sequential circuit is described in HW10, Problem 4. Assume the timing parameters for the gates and flip flops are as follows:

\begin{description}[noitemsep]
\item[Inverter:] $t_{pd}$ = 0.05 ns
\item[AND gate:] $t_{pd}$ = 0.1 ns
\item[OR gate:] $t_{pd}$ = 0.1 ns
\item[Flip flop:] $t_{pd}$ = 0.2 ns, $t_{s}$ = 0.02 ns, $t_{h}$ = 0.01 ns
\item[Note:] $t_{pd}$ = propagation delay, $t_{s}$ = setup time, $t_{h}$ = hold time
\end{description}

\begin{enumerate} [noitemsep, label=\alph*)]
\item Find the longest path delay from an external circuit input passing through gates only to an external circuit output.
\item Find the longest path delay from an external circuit input to positive clock edge.
\item Find the longest path delay from a positive clock edge to output.
\item Find the longest path delay from positive clock edge to positive clock edge.
\item Determine the maximum frequency of operation of the circuit in GHz.
\end{enumerate}

\vspace{\fill}
\noindent
GRADING SCALE
\medskip

Total: 31 pts
\bigskip

\def\arraystretch{1.5}
\begin{tabular}{ | l | c | c | c | c | c | c | c | c | } \hline
Pts          & 0  & 4  & 8  & 12 & 16 & 19 & 23 & 27     \\\hline
Letter Grade & D- & D  & C- & C  & B- & B  & A- & A      \\\hline
\end{tabular}
\end{raggedright}
\end{document}
