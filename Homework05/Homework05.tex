\documentclass[12pt,letterpaper,titlepage]{article}

\usepackage{fontspec}
\defaultfontfeatures{Mapping=tex-text}
\usepackage{xunicode}
\usepackage{xltxtra}
\usepackage{amsmath}
\usepackage{pdfpages}
\usepackage{amsfonts}
\usepackage{amssymb}
\setcounter{secnumdepth}{0}
\usepackage{nameref}
\usepackage{enumitem}
\usepackage{environ}

\setmainfont{Times New Roman}
\showboxdepth=\maxdimen
\showboxbreadth=\maxdimen


\usepackage{paracol}
\usepackage{wrapfig}
\globalcounter{table}
\globalcounter{figure}
\usepackage{graphicx}
\usepackage[left=1in,right=1in,top=1in,bottom=1in]{geometry}
\graphicspath{{img/}}

\author{Jacob Abel}
\title{	Homework 5
	\\\large ECE2004 CRN:12898
}

\setlength{\parskip}{0.5em}

\begin{document}
\maketitle
\begin{raggedright}

\paragraph{Question 1: }


\begin{center}
\includegraphics[width=.75\textwidth, height=\textheight, keepaspectratio=true]{hw5q1}
\end{center}

\begin{enumerate}[label=\Alph*)]
\item Solve for $V_{out}$.

$V_{out} = - \frac{300\Omega}{30\Omega}(1V+0.5V+0.5V)=-20V$

\item What type of amplifier circuit is this?

This is an inverting summing amplifier.

\end{enumerate}


\clearpage

\paragraph{Question 2: }

\begin{center}
\includegraphics[width=.75\textwidth, height=\textheight, keepaspectratio=true]{hw5q2}
\end{center}


\begin{enumerate}[label=\Alph*)]
\item Solve for the gain of this circuit, $H=\frac{V_{out}}{V_{in}}$.

$H = - \frac{1k\Omega}{1k\Omega} = -1$

\item What type of circuit is this?

This is an inverting amplifier.

\end{enumerate}

\clearpage

\paragraph{Question 3: }

\begin{center}
\includegraphics[width=.5\textwidth, height=\textheight, keepaspectratio=true]{hw5q3}
\end{center}

\begin{enumerate}[label=\Alph*)]
\item Solve for $V_{out}$.

$V_{out} = (1 + \frac{500\Omega}{500\Omega})20V = 40V$

\item What type of amplifier circuit is this?

This is a non--inverting amplifier.

\end{enumerate}


\clearpage

\paragraph{Question 4: }
Solve for $V_{out}$.

\begin{center}
\includegraphics[width=\textwidth, height=\textheight, keepaspectratio=true]{hw5q4}
\end{center}

\begin{align*}
\\ V_1     &= \frac{150\Omega}{15\Omega}(1V+0.5V+0.5V)=-20V
\\ V_2     &= - \frac{500\Omega}{500\Omega} \times -20V = 20V
\\ V_{out} &= (1 + \frac{500\Omega}{500\Omega})20V = 40V
\end{align*}

\end{raggedright}
\end{document}
