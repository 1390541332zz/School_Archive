\documentclass[12pt,letterpaper,titlepage]{article}

\usepackage{fontspec}
\defaultfontfeatures{Mapping=tex-text}
\usepackage{xunicode}
\usepackage{xltxtra}
\usepackage{amsmath}
\usepackage{pdfpages}
\usepackage{amsfonts}
\usepackage{bbold}
\usepackage{amssymb}
\setcounter{secnumdepth}{0}
\usepackage{nameref}
\usepackage{enumitem}
\usepackage{environ}
\usepackage{pgfplots}
\usepackage{listings}

\showboxdepth=\maxdimen
\showboxbreadth=\maxdimen


\usepackage{paracol}
\usepackage{wrapfig}
\globalcounter{table}
\globalcounter{figure}
\usepackage{graphicx}
\usepackage[left=1in,right=1in,top=1in,bottom=1in]{geometry}
\graphicspath{{img/}}

\author{Jacob Abel}
\title{	Homework 5
	\\\large ECE2500 CRN:82943
}

\setlength{\parskip}{0.5em}

\begin{document}
\maketitle
\begin{raggedright}

\section{Single Cycle Data path}
\begin{center}
\includegraphics[width=\textwidth, height=\textheight, keepaspectratio=true]{single}

Implements \texttt{add, sub, beq, lw, sw, j} instructions
\end{center}

\clearpage

\paragraph{Problem 1: }


\clearpage
\paragraph{Problem 2: }

\clearpage
\paragraph{Problem 3: }
Suppose your code contains a beq instruction in a loop which results in the following branch outcomes: T, NT, T, T, T, NT, T, NT, T, T, where T means taken and NT means not taken. (a) What is the accuracy of a static branch predictor that always predicts that the branch is taken? (b) What will be the accuracy of a 1-bit branch predictor, assuming it starts with a T prediction? (c) What will be the accuracy of a two-bit predictor assuming that the predictor starts off in the strongly taken state?


\end{raggedright}
\end{document}