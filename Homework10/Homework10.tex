\documentclass[12pt,letterpaper,titlepage]{report}
\usepackage{fontspec}
\defaultfontfeatures{Mapping=tex-text}
\usepackage{xunicode}
\usepackage{xltxtra}
\usepackage{enumitem}
\setmainfont{Times New Roman}
\usepackage{amsmath}
\usepackage{amsfonts}
\usepackage{amssymb}
\usepackage{multicol}
\usepackage{paracol}
\usepackage{multirow}
\usepackage{tikz}
\usepackage{graphicx}
\usepackage{tikz-timing}
\usetikzlibrary{automata,positioning}
\graphicspath{{img/}}
\usepackage{karnaugh-map}
\usepackage[margin=0.65in]{geometry}
\usepackage{xparse}
\author{Jacob Abel}
\title{%
	Homework 10
	\\\large ECE2504 CRN:82729
}



\begin{document}
\maketitle
\begin{raggedright}
\raggedcolumns

\paragraph{Note to the Grader}

Sorry I ran out of time, I wanted to actually do this HW but oh well.


\paragraph{Question 1:}
(3 pts) Clock, and S, R waveforms are shown below for a negative edge triggered SR flip flop. Sketch the output Q, obtained in response to the input waveforms. Assume that the propagation delay is negligible. The initial state is 0.

\includegraphics[width=0.25\textwidth, height=\textheight, keepaspectratio=true]{hw10p1a}

\begin{tikztimingtable}[%
    timing/slope=0,
    timing/.style={x=10ex,y=4ex},
    x=10ex,
    timing/rowdist=5ex,
    timing/name/.style={font=\sffamily\scriptsize}
]
CLK            & 15{c} \\
S              & 2.25h 6l 2h 4.75l\\
R              & 4.25l 4h 4l 2.25h\\
Q              & 2l 4h 4l 4h 1l\\
$\overline{Q}$ & 2h 4l 4h 4l 1h\\
\extracode
\begin{pgfonlayer}{background}
\begin{scope}[semitransparent ,semithick]
\vertlines[darkgray,dotted]{0.5,1.0 ,...,8.0}
\end{scope}
\end{pgfonlayer}
\end{tikztimingtable}

\paragraph{Question 2:}
(3 pts) Clock, S, R and clear waveforms are shown below for a negative edge triggered SR flip flop. Sketch the output Q, obtained in response to the input waveforms. Assume that the propagation delay is negligible. The initial state is 0.

\includegraphics[width=0.25\textwidth, height=\textheight, keepaspectratio=true]{hw10p2a}

\begin{tikztimingtable}[%
    timing/slope=0,
    timing/.style={x=10ex,y=4ex},
    x=10ex,
    timing/rowdist=5ex,
    timing/name/.style={font=\sffamily\scriptsize}
]
CLK            & 15{c} \\
S              & 2.25h 6l 2h 4.75l\\
R              & 4.25l 4h 4l 2.75h\\
CLR\_L         & 0.5h 2l 12.5h\\
Q              & 10l 4h 1l\\
$\overline{Q}$ & 10h 4l 1h\\
\extracode
\begin{pgfonlayer}{background}
\begin{scope}[semitransparent ,semithick]
\vertlines[darkgray,dotted]{0.5,1.0 ,...,8.0}
\end{scope}
\end{pgfonlayer}
\end{tikztimingtable}

\clearpage

\paragraph{Question 3:}
(3 pts) Clock and D waveforms are shown below for a positive edge triggered D flip flop shown. Sketch the output Q3, obtained is response to the input waveforms. Assume that the propagation delay is negligible. The initial state is 0.

\begin{center}
\includegraphics[width=0.75\textwidth, height=\textheight, keepaspectratio=true]{hw10p3a}
\end{center}

\begin{tikztimingtable}[%
    timing/slope=0,
    timing/.style={x=12.5ex,y=4ex},
    x=12.5ex,
    timing/rowdist=5ex,
    timing/name/.style={font=\sffamily\scriptsize}
]
CLK              & 11{c} \\
D                & 0.5l 1.75h 1l 1h 0.375l 1h 3l 1.5h 0.85l\\\
$Q_3$            & 1l 2h 2l 2h 2l 2h\\
$\overline{Q_3}$ & 1h 2l 2h 2l 2h 2l\\
\extracode
\begin{pgfonlayer}{background}
\begin{scope}[semitransparent ,semithick]
\vertlines[darkgray,dotted]{0.5,1.0,...,5.5}
\end{scope}
\end{pgfonlayer}
\end{tikztimingtable}

\clearpage

\paragraph{Question 4:}
(9 pts) A sequential circuit with two D flip flops A and B, two inputs X and Y, and one output Z is specified by the following input equations.

\begin{align*}
D_A&=X'Y+XB'\\
D_B&=XB+X'A\\
Z&=X'B
\end{align*}

\begin{enumerate} [noitemsep, label=\alph*)]
\item Draw the logic diagram.
\begin{center}
\includegraphics[width=0.9\textwidth, height=\textheight, keepaspectratio=true]{hw10p4a}
\end{center}

\pagebreak 

\item Derive the state table.
\begin{center}
\def\arraystretch{1.5}
\begin{tabular}{|c|c|c|c|c|c|}\hline 
\multicolumn{2}{|c|}{Current State} & Input & \multicolumn{2}{|c|}{Next State} & Output \\\hline 
$A$ & $B$ & XY & $A$ & $B$ & Z \\\hline 
0 & 0 & 00 & 0 & 0 & 0 \\\hline 
0 & 0 & 01 & 1 & 0 & 0 \\\hline 
0 & 0 & 10 & 1 & 0 & 0 \\\hline 
0 & 0 & 11 & 1 & 0 & 0 \\\hline 
0 & 1 & 00 & 0 & 0 & 1 \\\hline 
0 & 1 & 01 & 1 & 0 & 1 \\\hline 
0 & 1 & 10 & 0 & 1 & 0 \\\hline 
0 & 1 & 11 & 0 & 1 & 0 \\\hline 
1 & 0 & 00 & 0 & 0 & 0 \\\hline 
1 & 0 & 01 & 1 & 0 & 0 \\\hline 
1 & 0 & 10 & 1 & 1 & 0 \\\hline 
1 & 0 & 11 & 1 & 1 & 0 \\\hline 
1 & 1 & 00 & 0 & 1 & 1 \\\hline 
1 & 1 & 01 & 1 & 0 & 1 \\\hline 
1 & 1 & 10 & 0 & 0 & 0 \\\hline 
1 & 1 & 11 & 0 & 0 & 0 \\\hline 
\end{tabular} 
\end{center}

\item Derive the state diagram.
\end{enumerate}

\clearpage

\paragraph{Question 5:}
(10 pts) Starting from state 10 in the state diagram below, determine the state transitions and output sequence that will be generated when an input sequence of \textcolor{red}{1}01100111\textcolor{blue}{1} is applied. (Note that the 1 in red text is the first input bit applied, the 1 in blue text in the last input bit applied.)

\begin{center}
\includegraphics[width=0.375\textwidth, height=\textheight, keepaspectratio=true]{hw10p5a}
\end{center}

\clearpage

\paragraph{Question 6:}
Consider the following “divide‐by‐5” counter. (Don’t worry about the label “divide‐by‐5”; just look at how the flip flops are connected.)

\begin{center}
\includegraphics[width=0.75\textwidth, height=\textheight, keepaspectratio=true]{hw10p6a}
\end{center}

\begin{enumerate} [noitemsep, label=\alph*)]
\item (8) Complete the timing diagram beginning at a state of ABC=000

\begin{tikztimingtable}[%
    timing/slope=0,
    timing/.style={x=6.5ex,y=4ex},
    x=6.5ex,
    timing/rowdist=5ex,
    timing/name/.style={font=\sffamily\scriptsize}
]
CLK            & 21{c} \\
A              & \\
B              &  \\
C              &  \\
$J_A$          &   \\
$K_A$          &  \\
$J_B=K_B$      &  \\
$J_C$          & \\
$K_C$          &  \\
\extracode
\begin{pgfonlayer}{background}
\begin{scope}[semitransparent ,semithick]
\vertlines[darkgray,dotted]{0.5,1.5,...,9.5}
\end{scope}
\end{pgfonlayer}
\end{tikztimingtable}


\item (3) Draw the state diagram
\item (1) If the input clock frequency is 10 kHz, what is the frequency of the C signal?
\end{enumerate}

\clearpage

\paragraph{Question 7:}
(16 pts) Design a sequential circuit with two SR flip flops A and B and one input X. When X = 0, the state of the circuit remains the same. When X=1, the circuit goes through the state transitions from 00 to 10 to 11 to 01, back to 00 , then repeats.

\begin{enumerate} [noitemsep, label=\alph*)]
\item Show your state diagram or state table, K‐maps, FF input equations, and the logic diagram.
\item Repeat for JK flip flops
\end{enumerate}

\clearpage

\paragraph{Question 8:}
The state diagram for a sequential circuit in shown below. 

\begin{center}
\includegraphics[width=0.5\textwidth, height=\textheight, keepaspectratio=true]{hw10p8a}
\end{center}


\begin{enumerate} [noitemsep, label=\alph*)]
\item (4 pts) Find the state table
\item (1 pt) Make a state assignment
\item (3 pts) Find an optimized circuit implementation using D FFs, NAND gates, and inverters.
\end{enumerate}


\vspace{\fill}
\noindent
GRADING SCALE
\medskip

Total: 64 pts
\bigskip

\def\arraystretch{1.5} 
\begin{tabular}{ | l | c | c | c | c | c | c | c | c | } \hline
Pts          & 0  & 8  & 16 & 24 & 32 & 40 & 48 & 56     \\\hline
Letter Grade & D- & D  & C- & C  & B- & B  & A- & A      \\\hline
\end{tabular}
\end{raggedright}
\end{document}