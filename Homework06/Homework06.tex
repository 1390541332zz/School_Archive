\documentclass[12pt,letterpaper,titlepage]{article}
\usepackage{solarized-light}
\usepackage{fontspec}
\defaultfontfeatures{Mapping=tex-text}
\usepackage{xunicode}
\usepackage{xltxtra}
\usepackage{amsmath}
\usepackage{pdfpages}
\usepackage{amsfonts}
\usepackage{amssymb}
\setcounter{secnumdepth}{0}
\usepackage{nameref}
\usepackage{enumitem}
\usepackage{environ}
\usepackage{pgfplots}
\usepackage{karnaugh-map}

\setmainfont{Times New Roman}
\showboxdepth=\maxdimen
\showboxbreadth=\maxdimen


\usepackage{paracol}
\usepackage{wrapfig}
\globalcounter{table}
\globalcounter{figure}
\usepackage{graphicx}
\usepackage[left=1in,right=1in,top=1in,bottom=1in]{geometry}
\graphicspath{{img/}}

\author{Jacob Abel}
\title{	Homework 6
	\\\large ECE3544 CRN:82989
}

\setlength{\parskip}{0.25em}

\begin{document}
\maketitle
\begin{raggedright} 
\paragraph{Problem 1: }
For each of the following memory sizes, given as number of words x size of word in bits,
list the number of address lines and data lines are required.
\begin{enumerate}
\item 1K x 8: 10 addr lines, 7 data lines
\item 4K x 16: 12 addr lines, 16 data lines
\item 512 x 13: 9 addr lines, 13 data lines
\end{enumerate}

\paragraph{Problem 2: }
How many memory chips would be required to build a memory that is 8192 x 8, if each
memory chip is 256 x 1 bits? 32

\paragraph{Problem 3: }
Using the dual\_port\_rom module from slide 12 of the memory lecture as a starting point,
create a model of a single port ROM (with ports similar to the one on slide 10) and then use the
ROM to implement the sum and carry out for a 2-bit adder with carry in. The inputs will be
A[1:0], B[1:0], and c\_in, while outputs will be sum[1:0] and c\_out. Your ROM should be the
minimum size necessary to implement the adder. The ROM values should be stored in a text file
that is read in using the \$readmemh task. Create a testbench to show the correct operation of
your ROM's functionality.

Files included

\paragraph{Problem 4: }
Using the single\_clk\_ram module from slide 29 of the memory lecture as a starting point,
create a 16 x 8 memory where each memory location is initialized to (A5)16. Create a test bench
for the memory that writes the value x+1 to each memory address x (i.e., address 0 has value 1,
address 1 has value 2, address 2 has value 3, and so on), and then reads from each location to
verify that it has been written correctly. Your test bench should write all 16 locations before
reading them.


\end{raggedright}
\end{document}
