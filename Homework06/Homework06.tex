\documentclass[12pt,letterpaper,titlepage]{article}

\usepackage{fontspec}
\defaultfontfeatures{Mapping=tex-text}
\usepackage{xunicode}
\usepackage{xltxtra}
\usepackage{amsmath}
\usepackage{pdfpages}
\usepackage{mathtools}
\usepackage{amsfonts}
\usepackage{amssymb}
\setcounter{secnumdepth}{0}
\usepackage{nameref}
\usepackage{enumitem}
\usepackage{environ}

\showboxdepth=\maxdimen
\showboxbreadth=\maxdimen

\usepackage[tocflat]{tocstyle}
\usetocstyle{allwithdot}
\usepackage[bottom]{footmisc}

\usepackage{paracol}
\usepackage{wrapfig}
\globalcounter{table}
\globalcounter{figure}
\usepackage{graphicx}
\usepackage[left=1in,right=1in,top=1in,bottom=1in]{geometry}
\graphicspath{{img/}}

\author{Jacob Abel}
\title{	Homework 6
	\\\large MATH2534 CRN:15708
}

\setlength{\parskip}{0.5em}

\begin{document}
\maketitle
\begin{raggedright}
Prove by mathematical induction.
\begin{enumerate}[label=(\alph*)]

\item
$\sum_{i=1}^{n}(3i-1)=\frac{n(3n+1)}{2}$

Let $\forall n \in \mathbb{Z}, n \geq 1$

$\sum_{i=1}^{1}(3i-1)=\frac{n(3n+1)}{2}$

$3-1=\frac{3+1}{2} = 2$

The base case $P(1)$ when $n = 1$ is true.

$P(k)$ is $\sum_{i=1}^{k}(3i-1)=\frac{(k)(3(k)+1)}{2}$

$\forall k \in \mathbb{Z}, k \geq 1$, if $P(k)$ is true, then $P(k+1)$ is true.

$\sum_{i=1}^{k+1}(3i-1)=\frac{(k+1)(3(k+1)+1)}{2}$

$(3k+2)+\frac{3k^2+k}{2}=\frac{3k^2+k+6k+4}{2}=\frac{3k^2+7k+4}{2}$

By mathematical induction, the statement $\sum_{i=1}^{n}(3i-1)=\frac{n(3n+1)}{2}$ is true for all $n$.

\item
$1+\frac{1}{4}+\frac{1}{9}+\ldots+\frac{1}{n^2}\leq 2 - \frac{1}{n}$

Let $\forall n \in \mathbb{Z}, n \geq 1$

$\frac{1}{1^2} = 1 \leq 2 - \frac{1}{1} = 1$

The base case $P(1)$ when $n = 1$ is true.

$P(k)$ is $1+\frac{1}{4}+\frac{1}{9}+\ldots+\frac{1}{k^2}\leq 2 - \frac{1}{k}$

$\forall k \in \mathbb{Z}, k \geq 1$, if $P(k)$ is true, then $P(k+1)$ is true.

$1+\frac{1}{4}+\frac{1}{9}+\ldots+\frac{1}{k^2}+\frac{1}{(k+1)^2}\leq 2 - \frac{1}{k+1}$

$2 - \frac{1}{k}+\frac{1}{k^2+2k+1}\leq 2 - \frac{1}{k+1}$

$-\frac{1 + k + k^2}{k (1 + k)^2}\leq - \frac{1}{k+1}$


By mathematical induction, the statement $1+\frac{1}{4}+\frac{1}{9}+\ldots+\frac{1}{n^2}\leq 2 - \frac{1}{n}$ is true for all $n$.


\item
$(1-\frac{1}{2})(1-\frac{1}{4})\ldots(1-\frac{1}{2^n})\geq \frac{1}{4} + \frac{1}{2^{n+1}}$

---Did not complete---

\pagebreak
\item
For any integers $n \geq 7, n!>3^n$.

Let $\forall n \in \mathbb{Z}, n \geq 7$

$7! = 5040 > 3^7 = 2187$

The base case $P(7)$ when $n = 7$ is true.

$P(k)$ is $k!>3^k$

$\forall k \in \mathbb{Z}, k \geq 1$, if $P(k)$ is true, then $P(k+1)$ is true.

$(k+1)! = (k+1)\times k! > 3^{k+1} = (3)\times 3^k$

---Did not complete---

\item
For any integer $n \geq 1, 17^n-12^n$ is divisible by $5$.

Let $\forall n \in \mathbb{Z}, n \geq 1$

$17^1-12^1 = 17-12=5\implies 5|5$

The base case $P(1)$ when $n = 1$ is true.

$P(k)$ is $5 | (17^n-12^n)$

$\forall k \in \mathbb{Z}, k \geq 1$, if $P(k)$ is true, then $P(k+1)$ is true.

$17^{k+1}-12^{k+1} = 17\times17^k-12\times12^k$

$17^k-12^k$ is divisible by 5.

$17\times17^k-12\times12^k - 2(17^k-12^k) = 15\times17^k-10\times12^k$

$15\times17^k-10\times12^k$ is divisible by 5.

$17\times17^k-12\times12^k = 2(17^k-12^k) + (15\times17^k-10\times12^k)$

The sum of two terms divisible by 5 is also divisible by 5.

Therefore $17^{k+1}-12^{k+1}$ is divisible by 5.

By mathematical induction, the statement $1+\frac{1}{4}+\frac{1}{9}+\ldots+\frac{1}{n^2}\leq 2 - \frac{1}{n}$ is true for all $n$.

\end{enumerate}
\end{raggedright}
\end{document}
