\documentclass[12pt,letterpaper,titlepage]{article}

\usepackage{fontspec}
\defaultfontfeatures{Mapping=tex-text}
\usepackage{xunicode}
\usepackage{xltxtra}
\usepackage{amsmath}
\usepackage{pdfpages}
\usepackage{amsfonts}
\usepackage{bbold}
\usepackage{amssymb}
\setcounter{secnumdepth}{0}
\usepackage{nameref}
\usepackage{enumitem}
\usepackage{environ}
\usepackage{pgfplots}
\usepackage{listings}

\showboxdepth=\maxdimen
\showboxbreadth=\maxdimen


\usepackage{paracol}
\usepackage{wrapfig}
\globalcounter{table}
\globalcounter{figure}
\usepackage{graphicx}
\usepackage[left=1in,right=1in,top=1in,bottom=1in]{geometry}
\graphicspath{{img/}}

\author{Jacob Abel}
\title{	Homework 1
	\\\large ECE2500 CRN:82943
}

\setlength{\parskip}{0.5em}

\begin{document}
\maketitle
\begin{raggedright}

\paragraph{Problem 1: } There are many different examples of Instruction Set Architectures (ISAs).Give three examples of ISAs other than MIPS. For each example, give the following details:

\begin{itemize}
\item What differentiates the ISA from MIPS?
\item What similarities does the ISA share with MIPS?
\item What devices or applications most often use the ISA?
\end{itemize}

\begin{enumerate}
\item x86
\begin{itemize}
\item Unlike MIPS, x86 is a CISC architecture and therefore is more expensive to implement. x86 also has a variable instruction width unlike MIPS. Also x86 is little endian while MIPS can be either big or little endian.
\item Both MIPS and x86 are register based ISAs.
\item Most modern laptops and desktops are x86.
\end{itemize}

\item ARM
\begin{itemize}
\item ARM often implements branch prediction and has an extremely varied/heterogeneous architecture unlike MIPS which despite allowing extensions, tends to be much more homogeneous. 
\item Both ARM and MIPS are RISC ISAs, have fixed instruction widths, and are bi-endian.
\item Mobile devices and IoT devices often use ARM.
\end{itemize}

\item z/Architecture
\begin{itemize}
\item z/Architecture is primarily big endian, is CISC instead of RISC, and has variable width instructions unlike MIPS.
\item MIPS and z/Architecture are largely dissimilar except that they are both register based architectures.
\item This ISA is primarily used on IBM mainframes.
\end{itemize}

\end{enumerate}
\clearpage
\paragraph{Problem 2: }
Assume that the base address of the array, $A$, is stored in $\$s0$. The following table gives the memory addresses of $A[0], \ldots ,A[3]$ along with the values stored in the corresponding memory location.
\begin{center}
\begin{tabular}{|c|c|c|}
\hline
       & Stored Value 	& Memory Address 	\\\hline 
$A[0]$ & $-3$ 			&  $4$ 				\\\hline 
$A[1]$ &  $7$ 			&  $8$ 				\\\hline 
$A[2]$ & $-6$ 			& $12$ 				\\\hline 
$A[3]$ & $11$ 			& $16$ 				\\\hline 
\end{tabular} 
\end{center}
What are the values stored in registers $\$s0$, $\$s1$, $\$s2$, and $\$s3$ after executing the following MIPS instructions?
\columnratio{0.4}
\begin{paracol}{2}
\centering
Instructions
\begin{lstlisting}
sw  $s0, 0($s0) 
lw  $s1, 0($s0)
lw  $s2, 8($s0)
add $s3, $s1, $s2
\end{lstlisting}
\switchcolumn
\centering
Changed Values
\begin{lstlisting}
A[0]=4, $s0=4
A[0]=4, $s0=4, $s1=4
A[0]=4, $s0=4, $s1=4, $s2=-6
A[0]=4, $s0=4, $s1=4, $s2=-6, $s3=-2
\end{lstlisting}
\end{paracol}

Final Register Values: $\$s0=4,\, \$s1=4,\, \$s2=-6,\, \$s3=-2$

\clearpage
\paragraph{Problem 3: }
Convert each of the following C statements into MIPS instructions. Assume that the variables $a$, $b$, and $c$ are stored in $\$s0$, $\$s1$, and $\$s2$, respectively. The base addresses of the arrays $A$ and $B$ are stored in $\$s3$ and $\$s4$, respectively. Assume the arrays $A$ and $B$ contain integers.
\begin{enumerate}
\item $b = c - B[2^{17}]$
\begin{lstlisting}
lui   $s1, 0x0008
addu  $s1, $s1, $s4
lw    $s1, 0($s1)
sub   $s1, $s2, $s1
\end{lstlisting}

\item $A[3] = b + a + 2^{11}$
\begin{lstlisting}
addi  $t0, $s0, 0x0800
add   $t0, $t0, $s1
sw    $t0, 12($s3)
\end{lstlisting}

\item $B[5] = A[6] - c + B[b]$
\begin{lstlisting}
lw    $t0, 24($s3)
sll   $t1, $s1, 2
addu  $t1, $t1, $s4
lw    $t1, 0($t1)
sub   $t2, $t0, $s2
add   $t2, $t2, $t1
sw    $t2, 20($s4)
\end{lstlisting}

\item $A[1] = B[a - c] + b - c$
\begin{lstlisting}
sub   $t0, $s0, $s2
sll   $t0, $t0, 2
addu  $t0, $t0, $s4
lw    $t0, 0($t0)
add   $t0, $t0, $s1
sub   $t0, $t0, $s2
sw    $t0, 4($s3)
\end{lstlisting}

\end{enumerate}

\end{raggedright}
\end{document}
