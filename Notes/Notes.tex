\documentclass[12pt,letterpaper,titlepage]{article}
\usepackage{solarized-light}

\usepackage{fontspec}
\defaultfontfeatures{Mapping=tex-text}
\usepackage{xunicode}
\usepackage{xltxtra}
\usepackage{amsmath}
\usepackage{pdfpages}
\usepackage{amsfonts}
\usepackage{amssymb}
\setcounter{secnumdepth}{0}
\usepackage{nameref}

\setmainfont{Times New Roman}
\showboxdepth=\maxdimen
\showboxbreadth=\maxdimen

\usepackage[tocflat]{tocstyle}
\usetocstyle{allwithdot}
\usepackage[bottom]{footmisc}

\usepackage{karnaugh-map}
\usepackage{paracol}
\usepackage{wrapfig}
\globalcounter{table}
\globalcounter{figure}
\usepackage{graphicx}
\usepackage[left=1in,right=1in,top=1in,bottom=1in]{geometry}
\graphicspath{{img/}}

\author{Jacob Abel}
\title{ECE2004 Electric Circuit Analysis}

\setlength{\parskip}{0.5em}

\newcommand{\lstbg}[3][0pt]{{\fboxsep#1\colorbox{#2}{\strut #3}}}
\lstdefinelanguage{diff}{
  basicstyle=\ttfamily\small,
  morecomment=[f][\lstbg{red!20}]-,
  morecomment=[f][\lstbg{green!20}]+,
  morecomment=[f][\textit]{@@},
  %morecomment=[f][\textit]{---},
  %morecomment=[f][\textit]{+++},
}


\begin{document}
\maketitle


\tableofcontents
\pagebreak
\listoftables

\listoffigures

\pagebreak
\begin{raggedright}
  

  \section{Voltage \& Electric Potential}
  Electric Potential is more or less akin to gravitational potential.
  The negative of the battery is the lowest point of the potential.
  Series is different elevations from reference.
  Parallel is same elevation from reference.
  When 2 or more elements are connected between 2 points they are parallel.  

  \section{Passive Sign Convention}
  Signs and directions of current must be determined from the start. The direction is not particularly relevant as long as it is consistent.

  \section{Kirchhoff's Current Law}
  The sum of all currents at a node is always 0. 
  All currents must always balance out.
  \begin{equation}
    \sum _{k=1}^{n}{I}_{k}=0
  \end{equation}

  \section{Kirchhoff's Voltage Law}
  The sum of all voltages across a loop is always 0.
  \begin{equation}
    \sum _{k=1}^{n}V_{k}=0
  \end{equaion}
  KVL can be used to solve circuits using linear algebra.

  \section{Equivalent Resistance}
  You can sum up resistors in series or parallel to create an equivalent resistor that acts the same as all of its subcomponents.
  \begin{align}
    R_{eq}&=R_1+R+2
    R_{eq}&=
  \end{align}

\end{raggedright}
\end{document}
