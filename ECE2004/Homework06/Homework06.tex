\documentclass[12pt,letterpaper,titlepage]{article}

\usepackage{fontspec}
\defaultfontfeatures{Mapping=tex-text}
\usepackage{xunicode}
\usepackage{xltxtra}
\usepackage{amsmath}
\usepackage{pdfpages}
\usepackage{amsfonts}
\usepackage{amssymb}
\setcounter{secnumdepth}{0}
\usepackage{nameref}
\usepackage{enumitem}
\usepackage{environ}
\usepackage{pgfplots}

\setmainfont{Times New Roman}
\showboxdepth=\maxdimen
\showboxbreadth=\maxdimen


\usepackage{paracol}
\usepackage{wrapfig}
\globalcounter{table}
\globalcounter{figure}
\usepackage{graphicx}
\usepackage[left=1in,right=1in,top=1in,bottom=1in]{geometry}
\graphicspath{{img/}}

\author{Jacob Abel}
\title{	Homework 6
	\\\large ECE2004 CRN:12898
}

\setlength{\parskip}{0.5em}

\begin{document}
\maketitle
\begin{raggedright}

\paragraph{Problem 1: }
Assume the switch has been closed for a long time such that a steady state condition has been reached.

\begin{center}
\includegraphics[width=.75\textwidth, height=\textheight, keepaspectratio=true]{hw6q1}
\end{center}

\begin{enumerate}[label=\alph*)]
\item Find the time-domain formula for the voltage of the capacitor after the switch has been opened.

\begin{align*}
V_c &= \frac{500\Omega}{1k\Omega}10V = 5V
\\ q_c &= C \times V_0(1-e^{\frac{-t}{RC}})
\\     &= 1\mu F \times 5V(1-e^{\frac{-t}{500\Omega\times 1\mu F}})
\\     &= 5\mu C(1-e^{\frac{-t}{0.5ms}})
\end{align*}

\item What is total energy absorbed by the $500\Omega$ resistor after the switch has been opened for $(0 < t \leq \inf)$.

\begin{align*}
   E_r &= \frac{1}{2} C \times V^2
\\     &= \frac{1}{2} 1\mu F \times 25V^2
\\     &= 12.5\mu J
\end{align*}

\end{enumerate}


\clearpage

\paragraph{Problem 2: }
Assume switches have been in their position for a long enough time such that steady state conditions have been met and then they open/close at time equals zero.

\begin{center}
\includegraphics[width=.75\textwidth, height=\textheight, keepaspectratio=true]{hw6q2}
\end{center}


\begin{enumerate}[label=\alph*)]
\item What is the value of $i_L(0)$?

$i_L(0) = \frac{5V}{1k\Omega} = 5mA$

\item What is the value of $i_L(t)$ after $t = 0$?

\begin{align*}
i_L(t) &= i_L(0) e^{\frac{-Rt}{L}}
\\     &= 5mA\times e^{\frac{-1k\Omega\times t}{1mH}}
\\     &= 5mA\times e^{\frac{-t}{1\mu s}}
\end{align*}

\end{enumerate}

\clearpage

\paragraph{Problem 3: }
Assume that the capacitor has been charged to $10 V$ before a switch isolated the circuit below at time $t = 0$.

\begin{center}
\includegraphics[width=.5\textwidth, height=\textheight, keepaspectratio=true]{hw6q3}
\end{center}

\begin{enumerate}[label=\alph*)]
\item Plot the charge over time of the capacitor for this circuit, $q_c$.
\begin{align*}
   q_c &= C \times V_0(1-e^{\frac{-t}{RC}})
\\     &=  \times V_0(1-e^{\frac{-t}{RC}})
\end{align*}
\begin{figure}[ht]
\centering
\begin{tikzpicture}
\begin{axis}[
	title=Problem 3 RC Circuit Charge
	axis lines = left,
	xlabel = $Time\quad(ms)$, 
	ylabel = $Charge\quad(Q)$,
	legend style={empty legend},
	legend pos = outer north east]]
\addplot[
	domain=0:1,
	samples=100,
	red
]{10^-2*(1-e^(-x))};
\end{axis}
\end{tikzpicture}
\end{figure}

\item What is the time constant for this circuit? $\tau = 1ms$
\end{enumerate}


\clearpage

\paragraph{Problem 4: }
Assume the switch has been closed for a long time and is opened at $t = 0$. Find the current through the inductor after $t = 0$.
\begin{center}
\includegraphics[width=\textwidth, height=\textheight, keepaspectratio=true]{hw6q4}
\end{center}

\begin{align*}
V_0 &= 10V \times \frac{50\Omega}{(100\Omega\parallel 100\Omega) + 50\Omega} = 10V\frac{50\Omega}{100\Omega} = 5V
\\ i_L(0) &= \frac{5V}{50\Omega} = 0.1A
\\ i_L(t) &= i_L(0) e^{\frac{-Rt}{L}}
\\        &= 0.1A\times e^{\frac{-100\Omega \times t}{10mH}}
\\        &= 0.1A\times e^{\frac{-t}{100\mu s}}
\end{align*}


\end{raggedright}
\end{document}
