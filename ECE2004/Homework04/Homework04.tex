\documentclass[12pt,letterpaper,titlepage]{article}

\usepackage{fontspec}
\defaultfontfeatures{Mapping=tex-text}
\usepackage{xunicode}
\usepackage{xltxtra}
\usepackage{amsmath}
\usepackage{pdfpages}
\usepackage{amsfonts}
\usepackage{amssymb}
\setcounter{secnumdepth}{0}
\usepackage{nameref}
\usepackage{enumitem}
\usepackage{environ}

\setmainfont{Times New Roman}
\showboxdepth=\maxdimen
\showboxbreadth=\maxdimen


\usepackage{paracol}
\usepackage{wrapfig}
\globalcounter{table}
\globalcounter{figure}
\usepackage{graphicx}
\usepackage[left=1in,right=1in,top=1in,bottom=1in]{geometry}
\graphicspath{{img/}}

\author{Jacob Abel}
\title{	Homework 4
	\\\large ECE2004 CRN:12898
}

\setlength{\parskip}{0.5em}

\begin{document}
\maketitle
\begin{raggedright}

\paragraph{Question 1: }

Find the Thevenin equivalent voltage and resistance at terminals $a-b$.

\begin{center}
\includegraphics[width=.75\textwidth, height=\textheight, keepaspectratio=true]{hw4q1}
\end{center}

\begin{paracol}{2}
\begin{center}Closed\end{center}
\begin{align*}
    R_{5V}    &= 4k\Omega + 1k\Omega + 2.5k\Omega\parallel 5k\Omega = 6670\Omega
\\  I_{ab5V}  &= \frac{5V}{2.5k\Omega}
				 \times
				 \frac{2.5k\Omega\parallel 5k\Omega}{6670\Omega} = 0.5mA
\\  R_{2mA}   &= (4k\Omega + 1k\Omega)\parallel 5k\Omega \parallel 2.5k\Omega = 1250\Omega
\\  V_{ab2mA} &= 2mA \times 1250\Omega = 2.5V
\\  I_{ab2mA} &= \frac{2.5V}{2.5k\Omega} = 1mA
\\  I_{ab}    &= 1mA -0.5mA = 0.5mA
\end{align*}
\switchcolumn
\begin{center}Open\end{center}
\begin{align*}
    V_{ab5V}  &= V_{5k\Omega5V}
\\  V_{5k\Omega5V}  &= 5V\frac{5k\Omega}{5k\Omega + 4k\Omega + 1k\Omega} = 2.5V
\\  V_{ab2mA} &= V_{5k\Omega2mA}
\\  V_{5k\Omega2mA} &= 2mA\frac{1}{\frac{1}{5k\Omega}+\frac{1}{4k\Omega+1k\Omega}} = 5V
\\  V_{ab} &= 5V - 2.5V = 2.5V
\end{align*}
\end{paracol}
\begin{align*}
    V_{Th} &= 2.5V
\\  R_{Th} &= \frac{V_{ab}}{I_{ab}} = \frac{2.5V}{0.5mA} = 5k\Omega
\end{align*}
\clearpage

\paragraph{Question 2: }

Find the Norton equivalent current and resistance at terminals $a-b$.

\begin{center}
\includegraphics[width=\textwidth, height=\textheight, keepaspectratio=true]{hw4q2}
\end{center}


\clearpage

\paragraph{Question 3: }

What is the maximum power that can be delivered to $R_L$?

\begin{center}
\includegraphics[width=.75\textwidth, height=\textheight, keepaspectratio=true]{hw4q3}
\end{center}

\begin{paracol}{2}
\begin{center}Open\end{center}
\begin{align*}
    R_{12V}    &= 10\Omega + 5\Omega + 10\Omega = 25\Omega
\\  V_{R_L12V} &= V_{10\Omega12V}
\\  V_{10\Omega12V} &= 12V \frac{10\Omega}{25\Omega} = 4.8V
\\  R_{5A} &= 10\Omega\parallel(5\Omega + 10\Omega) = 6\Omega
\\  V_{R_L5A} &= V_{10\Omega5A}
\\  V_{10\Omega5A} &= 5A \times 6\Omega \frac{10\Omega}{5\Omega+10\Omega} = 20V
\\  V_{R_L} &= 20V + 4.8V = 24.8V
\end{align*}
\switchcolumn
\begin{center}Closed\end{center}
\begin{align*}
    R_{12V} &= 10\Omega + 5\Omega = 15\Omega
\\  I_{R_L12V} &= \frac{12V}{15\Omega} = 0.8A
\\  R_{5A} &= 10\Omega\parallel 5\Omega = \frac{10}{3}\Omega
\\  I_{R_L5A} &= 5A \frac{10\Omega}{15\Omega} = \frac{10}{3}A
\\  I_{R_L} &= 0.8A + \frac{10}{3}A = \frac{62}{15}A \approx 4.13A
\end{align*}
\end{paracol}
\begin{align*}
    V_{Th} &= 24.8V
\\  R_{Th} &= \frac{V_{ab}}{I_{ab}} = \frac{24.8V}{\frac{62}{15}A} = 6\Omega
\end{align*}

\clearpage

\paragraph{Question 4: }

\begin{center}
\includegraphics[width=0.75\textwidth, height=\textheight, keepaspectratio=true]{hw4q4}
\end{center}

\begin{enumerate}[label=\Alph*)]
\item Use superposition to find $V_x$.
\begin{paracol}{3}
\begin{align*}
    R_{20V}      &= 10\Omega  + 10\Omega = 20\Omega
\\  V_{x}        &= 20V \frac{10\Omega}{20\Omega} = 10V
\\  V_{10\Omega} &= 20V - 10V = 10V
\\  V_{20\Omega} &= 0V
\end{align*}
\switchcolumn
\begin{align*}
    R_{10V}      &= \frac{1}{\frac{1}{20\Omega} + \frac{1}{10\Omega + 10\Omega}} = 10\Omega
\\  V_{20\Omega} &= 10V
\\  V_{x}        &= 10V \frac{10\Omega}{20\Omega} = 5V
\\  V_{10\Omega} &= 10V - 5V = 5V
\end{align*}
\switchcolumn
\begin{align*}
    R_{3A}       &= \frac{1}{\frac{1}{10\Omega} + \frac{1}{10\Omega}} = 5\Omega
\\  V_{20\Omega} &= 0V
\\  V_{x}        &= 10\Omega\times3A\times\frac{1}{2} = 15V
\\  V_{10\Omega} &= 15V
\end{align*}
\end{paracol}
\begin{align*}
V_{x} &= 10V - 5V + 15V = 20V
\end{align*}
\item Analyse the full circuit with mesh current or nodal analysis to find $V_x$.



\end{enumerate}

\end{raggedright}
\end{document}
