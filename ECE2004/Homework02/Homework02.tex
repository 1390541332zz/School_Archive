\documentclass[12pt,letterpaper,titlepage]{article}

\usepackage{fontspec}
\defaultfontfeatures{Mapping=tex-text}
\usepackage{xunicode}
\usepackage{xltxtra}
\usepackage{amsmath}
\usepackage{pdfpages}
\usepackage{amsfonts}
\usepackage{amssymb}
\setcounter{secnumdepth}{0}
\usepackage{nameref}
\usepackage{enumitem}
\usepackage{environ}

\setmainfont{Times New Roman}
\showboxdepth=\maxdimen
\showboxbreadth=\maxdimen

\usepackage[tocflat]{tocstyle}
\usetocstyle{allwithdot}
\usepackage[bottom]{footmisc}

\usepackage{paracol}
\usepackage{wrapfig}
\globalcounter{table}
\globalcounter{figure}
\usepackage{graphicx}
\usepackage[left=1in,right=1in,top=1in,bottom=1in]{geometry}
\graphicspath{{img/}}

\author{Jacob Abel}
\title{	Homework 2
	\\\large ECE2004 CRN:12898
}

\setlength{\parskip}{0.5em}

\begin{document}
\maketitle

\begin{raggedright}

\paragraph{Question 1: }

Solve for $i$.

\begin{center}
\includegraphics[width=.75\textwidth, height=\textheight, keepaspectratio=true]{hw2q1}
\end{center}

\begin{align*}
   i &= \frac{30R}{R + (R \parallel ( R + (R \parallel R \parallel (R + ( R \parallel R)))))}
\\   &= \frac{30R}{R + (R \parallel ( R + (R \parallel R \parallel \frac{R+2R}{2})))}
\\   &= \frac{30R}{R + (R \parallel \frac{3R+8R}{8})}
\\   &= \frac{30R}{\frac{11R+19R}{19}}
\\   &= \frac{30R}{\frac{30R}{19}}
\\   &= \frac{30R\times19}{30R}
\\   &= 19A
\end{align*}

\clearpage

\paragraph{Question 2: }

If every resistor has a tolerance of $\pm 10\%$ and the nominal values are shown below, find the minimum and maximum current $i$ possible.

\begin{center}
\includegraphics[width=.75\textwidth, height=\textheight, keepaspectratio=true]{hw2q2}
\end{center}
\begin{align*}
    i &= \frac{5V}{4 k\Omega \pm 10\% + (2 k\Omega \pm 10\% \parallel (1 k\Omega \pm 10\% + 1 k\Omega \pm 10\%))}
\\    &= \frac{5V}{4 k\Omega \pm 10\% + (2 k\Omega \pm 10\% \parallel 2 k\Omega \pm 10\%)}
\\    &= \frac{5V}{4 k\Omega \pm 10\% + \frac{2 k\Omega \pm 10\%}{2}}
\\    &= \frac{5V}{4 k\Omega \pm 10\% + 1 k\Omega \pm 10\%}
\\    &= \frac{5V}{5 k\Omega \pm 10\%}
\\ i_{min} &= 0.\overline{90}A
\\ i_{max} &= 1.\overline{11}A
\end{align*}

\clearpage

\paragraph{Question 3: }

Find $V_x$.

\begin{center}
\includegraphics[width=\textwidth, height=\textheight, keepaspectratio=true]{hw2q3}
\end{center}

\begin{align*}
 V_x &= \frac{9A}{\frac{1}{3\Omega}+\frac{1}{3\Omega}+\frac{1}{3\Omega}+\frac{1}{6\Omega}+\frac{1}{6\Omega}+\frac{1}{6\Omega}}
\\   &= \frac{9A}{\frac{3}{3\Omega}+\frac{3}{6\Omega}}
\\   &= \frac{9A}{\frac{1}{1\Omega}+\frac{1}{2\Omega}}
\\   &= \frac{9A \times 2\Omega}{3}
\\   &= \frac{18}{3}V
\\   &= 6V
\end{align*}

\clearpage

\paragraph{Question 4: }

Find the power consumed by the $10\Omega$ resistor.

\begin{center}
\includegraphics[width=.75\textwidth, height=\textheight, keepaspectratio=true]{hw2q4}
\end{center}

\begin{align*}
    \frac{V_x-0}{7\Omega}-\frac{10V-V_x}{3\Omega}=&0
\\  \frac{10V_x}{21\Omega}=& \frac{10V}{3\Omega}
\\  V_x =& 7V
\\  \frac{10\times7V - 0}{10\Omega} =& 0
\\  i_{10\Omega} =& 7A
\\  P_{10\Omega} =& (7A)^2\times 10\Omega
\\   =& 490W
\end{align*}

\end{raggedright}
\end{document}
